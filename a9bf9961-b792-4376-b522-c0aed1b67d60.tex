
    




    
\documentclass[11pt]{article}

    
    \usepackage[breakable]{tcolorbox}
    \tcbset{nobeforeafter} % prevents tcolorboxes being placing in paragraphs
    \usepackage{float}
    \floatplacement{figure}{H} % forces figures to be placed at the correct location
    
    \usepackage[T1]{fontenc}
    % Nicer default font (+ math font) than Computer Modern for most use cases
    \usepackage{mathpazo}

    % Basic figure setup, for now with no caption control since it's done
    % automatically by Pandoc (which extracts ![](path) syntax from Markdown).
    \usepackage{graphicx}
    % We will generate all images so they have a width \maxwidth. This means
    % that they will get their normal width if they fit onto the page, but
    % are scaled down if they would overflow the margins.
    \makeatletter
    \def\maxwidth{\ifdim\Gin@nat@width>\linewidth\linewidth
    \else\Gin@nat@width\fi}
    \makeatother
    \let\Oldincludegraphics\includegraphics
    % Set max figure width to be 80% of text width, for now hardcoded.
    \renewcommand{\includegraphics}[1]{\Oldincludegraphics[width=.8\maxwidth]{#1}}
    % Ensure that by default, figures have no caption (until we provide a
    % proper Figure object with a Caption API and a way to capture that
    % in the conversion process - todo).
    \usepackage{caption}
    \DeclareCaptionLabelFormat{nolabel}{}
    \captionsetup{labelformat=nolabel}

    \usepackage{adjustbox} % Used to constrain images to a maximum size 
    \usepackage{xcolor} % Allow colors to be defined
    \usepackage{enumerate} % Needed for markdown enumerations to work
    \usepackage{geometry} % Used to adjust the document margins
    \usepackage{amsmath} % Equations
    \usepackage{amssymb} % Equations
    \usepackage{textcomp} % defines textquotesingle
    % Hack from http://tex.stackexchange.com/a/47451/13684:
    \AtBeginDocument{%
        \def\PYZsq{\textquotesingle}% Upright quotes in Pygmentized code
    }
    \usepackage{upquote} % Upright quotes for verbatim code
    \usepackage{eurosym} % defines \euro
    \usepackage[mathletters]{ucs} % Extended unicode (utf-8) support
    \usepackage[utf8x]{inputenc} % Allow utf-8 characters in the tex document
    \usepackage{fancyvrb} % verbatim replacement that allows latex
    \usepackage{grffile} % extends the file name processing of package graphics 
                         % to support a larger range 
    % The hyperref package gives us a pdf with properly built
    % internal navigation ('pdf bookmarks' for the table of contents,
    % internal cross-reference links, web links for URLs, etc.)
    \usepackage{hyperref}
    \usepackage{longtable} % longtable support required by pandoc >1.10
    \usepackage{booktabs}  % table support for pandoc > 1.12.2
    \usepackage[inline]{enumitem} % IRkernel/repr support (it uses the enumerate* environment)
    \usepackage[normalem]{ulem} % ulem is needed to support strikethroughs (\sout)
                                % normalem makes italics be italics, not underlines
    \usepackage{mathrsfs}
    

    
    % Colors for the hyperref package
    \definecolor{urlcolor}{rgb}{0,.145,.698}
    \definecolor{linkcolor}{rgb}{.71,0.21,0.01}
    \definecolor{citecolor}{rgb}{.12,.54,.11}

    % ANSI colors
    \definecolor{ansi-black}{HTML}{3E424D}
    \definecolor{ansi-black-intense}{HTML}{282C36}
    \definecolor{ansi-red}{HTML}{E75C58}
    \definecolor{ansi-red-intense}{HTML}{B22B31}
    \definecolor{ansi-green}{HTML}{00A250}
    \definecolor{ansi-green-intense}{HTML}{007427}
    \definecolor{ansi-yellow}{HTML}{DDB62B}
    \definecolor{ansi-yellow-intense}{HTML}{B27D12}
    \definecolor{ansi-blue}{HTML}{208FFB}
    \definecolor{ansi-blue-intense}{HTML}{0065CA}
    \definecolor{ansi-magenta}{HTML}{D160C4}
    \definecolor{ansi-magenta-intense}{HTML}{A03196}
    \definecolor{ansi-cyan}{HTML}{60C6C8}
    \definecolor{ansi-cyan-intense}{HTML}{258F8F}
    \definecolor{ansi-white}{HTML}{C5C1B4}
    \definecolor{ansi-white-intense}{HTML}{A1A6B2}
    \definecolor{ansi-default-inverse-fg}{HTML}{FFFFFF}
    \definecolor{ansi-default-inverse-bg}{HTML}{000000}

    % commands and environments needed by pandoc snippets
    % extracted from the output of `pandoc -s`
    \providecommand{\tightlist}{%
      \setlength{\itemsep}{0pt}\setlength{\parskip}{0pt}}
    \DefineVerbatimEnvironment{Highlighting}{Verbatim}{commandchars=\\\{\}}
    % Add ',fontsize=\small' for more characters per line
    \newenvironment{Shaded}{}{}
    \newcommand{\KeywordTok}[1]{\textcolor[rgb]{0.00,0.44,0.13}{\textbf{{#1}}}}
    \newcommand{\DataTypeTok}[1]{\textcolor[rgb]{0.56,0.13,0.00}{{#1}}}
    \newcommand{\DecValTok}[1]{\textcolor[rgb]{0.25,0.63,0.44}{{#1}}}
    \newcommand{\BaseNTok}[1]{\textcolor[rgb]{0.25,0.63,0.44}{{#1}}}
    \newcommand{\FloatTok}[1]{\textcolor[rgb]{0.25,0.63,0.44}{{#1}}}
    \newcommand{\CharTok}[1]{\textcolor[rgb]{0.25,0.44,0.63}{{#1}}}
    \newcommand{\StringTok}[1]{\textcolor[rgb]{0.25,0.44,0.63}{{#1}}}
    \newcommand{\CommentTok}[1]{\textcolor[rgb]{0.38,0.63,0.69}{\textit{{#1}}}}
    \newcommand{\OtherTok}[1]{\textcolor[rgb]{0.00,0.44,0.13}{{#1}}}
    \newcommand{\AlertTok}[1]{\textcolor[rgb]{1.00,0.00,0.00}{\textbf{{#1}}}}
    \newcommand{\FunctionTok}[1]{\textcolor[rgb]{0.02,0.16,0.49}{{#1}}}
    \newcommand{\RegionMarkerTok}[1]{{#1}}
    \newcommand{\ErrorTok}[1]{\textcolor[rgb]{1.00,0.00,0.00}{\textbf{{#1}}}}
    \newcommand{\NormalTok}[1]{{#1}}
    
    % Additional commands for more recent versions of Pandoc
    \newcommand{\ConstantTok}[1]{\textcolor[rgb]{0.53,0.00,0.00}{{#1}}}
    \newcommand{\SpecialCharTok}[1]{\textcolor[rgb]{0.25,0.44,0.63}{{#1}}}
    \newcommand{\VerbatimStringTok}[1]{\textcolor[rgb]{0.25,0.44,0.63}{{#1}}}
    \newcommand{\SpecialStringTok}[1]{\textcolor[rgb]{0.73,0.40,0.53}{{#1}}}
    \newcommand{\ImportTok}[1]{{#1}}
    \newcommand{\DocumentationTok}[1]{\textcolor[rgb]{0.73,0.13,0.13}{\textit{{#1}}}}
    \newcommand{\AnnotationTok}[1]{\textcolor[rgb]{0.38,0.63,0.69}{\textbf{\textit{{#1}}}}}
    \newcommand{\CommentVarTok}[1]{\textcolor[rgb]{0.38,0.63,0.69}{\textbf{\textit{{#1}}}}}
    \newcommand{\VariableTok}[1]{\textcolor[rgb]{0.10,0.09,0.49}{{#1}}}
    \newcommand{\ControlFlowTok}[1]{\textcolor[rgb]{0.00,0.44,0.13}{\textbf{{#1}}}}
    \newcommand{\OperatorTok}[1]{\textcolor[rgb]{0.40,0.40,0.40}{{#1}}}
    \newcommand{\BuiltInTok}[1]{{#1}}
    \newcommand{\ExtensionTok}[1]{{#1}}
    \newcommand{\PreprocessorTok}[1]{\textcolor[rgb]{0.74,0.48,0.00}{{#1}}}
    \newcommand{\AttributeTok}[1]{\textcolor[rgb]{0.49,0.56,0.16}{{#1}}}
    \newcommand{\InformationTok}[1]{\textcolor[rgb]{0.38,0.63,0.69}{\textbf{\textit{{#1}}}}}
    \newcommand{\WarningTok}[1]{\textcolor[rgb]{0.38,0.63,0.69}{\textbf{\textit{{#1}}}}}
    
    
    % Define a nice break command that doesn't care if a line doesn't already
    % exist.
    \def\br{\hspace*{\fill} \\* }
    % Math Jax compatibility definitions
    \def\gt{>}
    \def\lt{<}
    \let\Oldtex\TeX
    \let\Oldlatex\LaTeX
    \renewcommand{\TeX}{\textrm{\Oldtex}}
    \renewcommand{\LaTeX}{\textrm{\Oldlatex}}
    % Document parameters
    % Document title
    \title{a9bf9961-b792-4376-b522-c0aed1b67d60}
    
    
    
    
    
% Pygments definitions
\makeatletter
\def\PY@reset{\let\PY@it=\relax \let\PY@bf=\relax%
    \let\PY@ul=\relax \let\PY@tc=\relax%
    \let\PY@bc=\relax \let\PY@ff=\relax}
\def\PY@tok#1{\csname PY@tok@#1\endcsname}
\def\PY@toks#1+{\ifx\relax#1\empty\else%
    \PY@tok{#1}\expandafter\PY@toks\fi}
\def\PY@do#1{\PY@bc{\PY@tc{\PY@ul{%
    \PY@it{\PY@bf{\PY@ff{#1}}}}}}}
\def\PY#1#2{\PY@reset\PY@toks#1+\relax+\PY@do{#2}}

\expandafter\def\csname PY@tok@w\endcsname{\def\PY@tc##1{\textcolor[rgb]{0.73,0.73,0.73}{##1}}}
\expandafter\def\csname PY@tok@c\endcsname{\let\PY@it=\textit\def\PY@tc##1{\textcolor[rgb]{0.25,0.50,0.50}{##1}}}
\expandafter\def\csname PY@tok@cp\endcsname{\def\PY@tc##1{\textcolor[rgb]{0.74,0.48,0.00}{##1}}}
\expandafter\def\csname PY@tok@k\endcsname{\let\PY@bf=\textbf\def\PY@tc##1{\textcolor[rgb]{0.00,0.50,0.00}{##1}}}
\expandafter\def\csname PY@tok@kp\endcsname{\def\PY@tc##1{\textcolor[rgb]{0.00,0.50,0.00}{##1}}}
\expandafter\def\csname PY@tok@kt\endcsname{\def\PY@tc##1{\textcolor[rgb]{0.69,0.00,0.25}{##1}}}
\expandafter\def\csname PY@tok@o\endcsname{\def\PY@tc##1{\textcolor[rgb]{0.40,0.40,0.40}{##1}}}
\expandafter\def\csname PY@tok@ow\endcsname{\let\PY@bf=\textbf\def\PY@tc##1{\textcolor[rgb]{0.67,0.13,1.00}{##1}}}
\expandafter\def\csname PY@tok@nb\endcsname{\def\PY@tc##1{\textcolor[rgb]{0.00,0.50,0.00}{##1}}}
\expandafter\def\csname PY@tok@nf\endcsname{\def\PY@tc##1{\textcolor[rgb]{0.00,0.00,1.00}{##1}}}
\expandafter\def\csname PY@tok@nc\endcsname{\let\PY@bf=\textbf\def\PY@tc##1{\textcolor[rgb]{0.00,0.00,1.00}{##1}}}
\expandafter\def\csname PY@tok@nn\endcsname{\let\PY@bf=\textbf\def\PY@tc##1{\textcolor[rgb]{0.00,0.00,1.00}{##1}}}
\expandafter\def\csname PY@tok@ne\endcsname{\let\PY@bf=\textbf\def\PY@tc##1{\textcolor[rgb]{0.82,0.25,0.23}{##1}}}
\expandafter\def\csname PY@tok@nv\endcsname{\def\PY@tc##1{\textcolor[rgb]{0.10,0.09,0.49}{##1}}}
\expandafter\def\csname PY@tok@no\endcsname{\def\PY@tc##1{\textcolor[rgb]{0.53,0.00,0.00}{##1}}}
\expandafter\def\csname PY@tok@nl\endcsname{\def\PY@tc##1{\textcolor[rgb]{0.63,0.63,0.00}{##1}}}
\expandafter\def\csname PY@tok@ni\endcsname{\let\PY@bf=\textbf\def\PY@tc##1{\textcolor[rgb]{0.60,0.60,0.60}{##1}}}
\expandafter\def\csname PY@tok@na\endcsname{\def\PY@tc##1{\textcolor[rgb]{0.49,0.56,0.16}{##1}}}
\expandafter\def\csname PY@tok@nt\endcsname{\let\PY@bf=\textbf\def\PY@tc##1{\textcolor[rgb]{0.00,0.50,0.00}{##1}}}
\expandafter\def\csname PY@tok@nd\endcsname{\def\PY@tc##1{\textcolor[rgb]{0.67,0.13,1.00}{##1}}}
\expandafter\def\csname PY@tok@s\endcsname{\def\PY@tc##1{\textcolor[rgb]{0.73,0.13,0.13}{##1}}}
\expandafter\def\csname PY@tok@sd\endcsname{\let\PY@it=\textit\def\PY@tc##1{\textcolor[rgb]{0.73,0.13,0.13}{##1}}}
\expandafter\def\csname PY@tok@si\endcsname{\let\PY@bf=\textbf\def\PY@tc##1{\textcolor[rgb]{0.73,0.40,0.53}{##1}}}
\expandafter\def\csname PY@tok@se\endcsname{\let\PY@bf=\textbf\def\PY@tc##1{\textcolor[rgb]{0.73,0.40,0.13}{##1}}}
\expandafter\def\csname PY@tok@sr\endcsname{\def\PY@tc##1{\textcolor[rgb]{0.73,0.40,0.53}{##1}}}
\expandafter\def\csname PY@tok@ss\endcsname{\def\PY@tc##1{\textcolor[rgb]{0.10,0.09,0.49}{##1}}}
\expandafter\def\csname PY@tok@sx\endcsname{\def\PY@tc##1{\textcolor[rgb]{0.00,0.50,0.00}{##1}}}
\expandafter\def\csname PY@tok@m\endcsname{\def\PY@tc##1{\textcolor[rgb]{0.40,0.40,0.40}{##1}}}
\expandafter\def\csname PY@tok@gh\endcsname{\let\PY@bf=\textbf\def\PY@tc##1{\textcolor[rgb]{0.00,0.00,0.50}{##1}}}
\expandafter\def\csname PY@tok@gu\endcsname{\let\PY@bf=\textbf\def\PY@tc##1{\textcolor[rgb]{0.50,0.00,0.50}{##1}}}
\expandafter\def\csname PY@tok@gd\endcsname{\def\PY@tc##1{\textcolor[rgb]{0.63,0.00,0.00}{##1}}}
\expandafter\def\csname PY@tok@gi\endcsname{\def\PY@tc##1{\textcolor[rgb]{0.00,0.63,0.00}{##1}}}
\expandafter\def\csname PY@tok@gr\endcsname{\def\PY@tc##1{\textcolor[rgb]{1.00,0.00,0.00}{##1}}}
\expandafter\def\csname PY@tok@ge\endcsname{\let\PY@it=\textit}
\expandafter\def\csname PY@tok@gs\endcsname{\let\PY@bf=\textbf}
\expandafter\def\csname PY@tok@gp\endcsname{\let\PY@bf=\textbf\def\PY@tc##1{\textcolor[rgb]{0.00,0.00,0.50}{##1}}}
\expandafter\def\csname PY@tok@go\endcsname{\def\PY@tc##1{\textcolor[rgb]{0.53,0.53,0.53}{##1}}}
\expandafter\def\csname PY@tok@gt\endcsname{\def\PY@tc##1{\textcolor[rgb]{0.00,0.27,0.87}{##1}}}
\expandafter\def\csname PY@tok@err\endcsname{\def\PY@bc##1{\setlength{\fboxsep}{0pt}\fcolorbox[rgb]{1.00,0.00,0.00}{1,1,1}{\strut ##1}}}
\expandafter\def\csname PY@tok@kc\endcsname{\let\PY@bf=\textbf\def\PY@tc##1{\textcolor[rgb]{0.00,0.50,0.00}{##1}}}
\expandafter\def\csname PY@tok@kd\endcsname{\let\PY@bf=\textbf\def\PY@tc##1{\textcolor[rgb]{0.00,0.50,0.00}{##1}}}
\expandafter\def\csname PY@tok@kn\endcsname{\let\PY@bf=\textbf\def\PY@tc##1{\textcolor[rgb]{0.00,0.50,0.00}{##1}}}
\expandafter\def\csname PY@tok@kr\endcsname{\let\PY@bf=\textbf\def\PY@tc##1{\textcolor[rgb]{0.00,0.50,0.00}{##1}}}
\expandafter\def\csname PY@tok@bp\endcsname{\def\PY@tc##1{\textcolor[rgb]{0.00,0.50,0.00}{##1}}}
\expandafter\def\csname PY@tok@fm\endcsname{\def\PY@tc##1{\textcolor[rgb]{0.00,0.00,1.00}{##1}}}
\expandafter\def\csname PY@tok@vc\endcsname{\def\PY@tc##1{\textcolor[rgb]{0.10,0.09,0.49}{##1}}}
\expandafter\def\csname PY@tok@vg\endcsname{\def\PY@tc##1{\textcolor[rgb]{0.10,0.09,0.49}{##1}}}
\expandafter\def\csname PY@tok@vi\endcsname{\def\PY@tc##1{\textcolor[rgb]{0.10,0.09,0.49}{##1}}}
\expandafter\def\csname PY@tok@vm\endcsname{\def\PY@tc##1{\textcolor[rgb]{0.10,0.09,0.49}{##1}}}
\expandafter\def\csname PY@tok@sa\endcsname{\def\PY@tc##1{\textcolor[rgb]{0.73,0.13,0.13}{##1}}}
\expandafter\def\csname PY@tok@sb\endcsname{\def\PY@tc##1{\textcolor[rgb]{0.73,0.13,0.13}{##1}}}
\expandafter\def\csname PY@tok@sc\endcsname{\def\PY@tc##1{\textcolor[rgb]{0.73,0.13,0.13}{##1}}}
\expandafter\def\csname PY@tok@dl\endcsname{\def\PY@tc##1{\textcolor[rgb]{0.73,0.13,0.13}{##1}}}
\expandafter\def\csname PY@tok@s2\endcsname{\def\PY@tc##1{\textcolor[rgb]{0.73,0.13,0.13}{##1}}}
\expandafter\def\csname PY@tok@sh\endcsname{\def\PY@tc##1{\textcolor[rgb]{0.73,0.13,0.13}{##1}}}
\expandafter\def\csname PY@tok@s1\endcsname{\def\PY@tc##1{\textcolor[rgb]{0.73,0.13,0.13}{##1}}}
\expandafter\def\csname PY@tok@mb\endcsname{\def\PY@tc##1{\textcolor[rgb]{0.40,0.40,0.40}{##1}}}
\expandafter\def\csname PY@tok@mf\endcsname{\def\PY@tc##1{\textcolor[rgb]{0.40,0.40,0.40}{##1}}}
\expandafter\def\csname PY@tok@mh\endcsname{\def\PY@tc##1{\textcolor[rgb]{0.40,0.40,0.40}{##1}}}
\expandafter\def\csname PY@tok@mi\endcsname{\def\PY@tc##1{\textcolor[rgb]{0.40,0.40,0.40}{##1}}}
\expandafter\def\csname PY@tok@il\endcsname{\def\PY@tc##1{\textcolor[rgb]{0.40,0.40,0.40}{##1}}}
\expandafter\def\csname PY@tok@mo\endcsname{\def\PY@tc##1{\textcolor[rgb]{0.40,0.40,0.40}{##1}}}
\expandafter\def\csname PY@tok@ch\endcsname{\let\PY@it=\textit\def\PY@tc##1{\textcolor[rgb]{0.25,0.50,0.50}{##1}}}
\expandafter\def\csname PY@tok@cm\endcsname{\let\PY@it=\textit\def\PY@tc##1{\textcolor[rgb]{0.25,0.50,0.50}{##1}}}
\expandafter\def\csname PY@tok@cpf\endcsname{\let\PY@it=\textit\def\PY@tc##1{\textcolor[rgb]{0.25,0.50,0.50}{##1}}}
\expandafter\def\csname PY@tok@c1\endcsname{\let\PY@it=\textit\def\PY@tc##1{\textcolor[rgb]{0.25,0.50,0.50}{##1}}}
\expandafter\def\csname PY@tok@cs\endcsname{\let\PY@it=\textit\def\PY@tc##1{\textcolor[rgb]{0.25,0.50,0.50}{##1}}}

\def\PYZbs{\char`\\}
\def\PYZus{\char`\_}
\def\PYZob{\char`\{}
\def\PYZcb{\char`\}}
\def\PYZca{\char`\^}
\def\PYZam{\char`\&}
\def\PYZlt{\char`\<}
\def\PYZgt{\char`\>}
\def\PYZsh{\char`\#}
\def\PYZpc{\char`\%}
\def\PYZdl{\char`\$}
\def\PYZhy{\char`\-}
\def\PYZsq{\char`\'}
\def\PYZdq{\char`\"}
\def\PYZti{\char`\~}
% for compatibility with earlier versions
\def\PYZat{@}
\def\PYZlb{[}
\def\PYZrb{]}
\makeatother


    % For linebreaks inside Verbatim environment from package fancyvrb. 
    \makeatletter
        \newbox\Wrappedcontinuationbox 
        \newbox\Wrappedvisiblespacebox 
        \newcommand*\Wrappedvisiblespace {\textcolor{red}{\textvisiblespace}} 
        \newcommand*\Wrappedcontinuationsymbol {\textcolor{red}{\llap{\tiny$\m@th\hookrightarrow$}}} 
        \newcommand*\Wrappedcontinuationindent {3ex } 
        \newcommand*\Wrappedafterbreak {\kern\Wrappedcontinuationindent\copy\Wrappedcontinuationbox} 
        % Take advantage of the already applied Pygments mark-up to insert 
        % potential linebreaks for TeX processing. 
        %        {, <, #, %, $, ' and ": go to next line. 
        %        _, }, ^, &, >, - and ~: stay at end of broken line. 
        % Use of \textquotesingle for straight quote. 
        \newcommand*\Wrappedbreaksatspecials {% 
            \def\PYGZus{\discretionary{\char`\_}{\Wrappedafterbreak}{\char`\_}}% 
            \def\PYGZob{\discretionary{}{\Wrappedafterbreak\char`\{}{\char`\{}}% 
            \def\PYGZcb{\discretionary{\char`\}}{\Wrappedafterbreak}{\char`\}}}% 
            \def\PYGZca{\discretionary{\char`\^}{\Wrappedafterbreak}{\char`\^}}% 
            \def\PYGZam{\discretionary{\char`\&}{\Wrappedafterbreak}{\char`\&}}% 
            \def\PYGZlt{\discretionary{}{\Wrappedafterbreak\char`\<}{\char`\<}}% 
            \def\PYGZgt{\discretionary{\char`\>}{\Wrappedafterbreak}{\char`\>}}% 
            \def\PYGZsh{\discretionary{}{\Wrappedafterbreak\char`\#}{\char`\#}}% 
            \def\PYGZpc{\discretionary{}{\Wrappedafterbreak\char`\%}{\char`\%}}% 
            \def\PYGZdl{\discretionary{}{\Wrappedafterbreak\char`\$}{\char`\$}}% 
            \def\PYGZhy{\discretionary{\char`\-}{\Wrappedafterbreak}{\char`\-}}% 
            \def\PYGZsq{\discretionary{}{\Wrappedafterbreak\textquotesingle}{\textquotesingle}}% 
            \def\PYGZdq{\discretionary{}{\Wrappedafterbreak\char`\"}{\char`\"}}% 
            \def\PYGZti{\discretionary{\char`\~}{\Wrappedafterbreak}{\char`\~}}% 
        } 
        % Some characters . , ; ? ! / are not pygmentized. 
        % This macro makes them "active" and they will insert potential linebreaks 
        \newcommand*\Wrappedbreaksatpunct {% 
            \lccode`\~`\.\lowercase{\def~}{\discretionary{\hbox{\char`\.}}{\Wrappedafterbreak}{\hbox{\char`\.}}}% 
            \lccode`\~`\,\lowercase{\def~}{\discretionary{\hbox{\char`\,}}{\Wrappedafterbreak}{\hbox{\char`\,}}}% 
            \lccode`\~`\;\lowercase{\def~}{\discretionary{\hbox{\char`\;}}{\Wrappedafterbreak}{\hbox{\char`\;}}}% 
            \lccode`\~`\:\lowercase{\def~}{\discretionary{\hbox{\char`\:}}{\Wrappedafterbreak}{\hbox{\char`\:}}}% 
            \lccode`\~`\?\lowercase{\def~}{\discretionary{\hbox{\char`\?}}{\Wrappedafterbreak}{\hbox{\char`\?}}}% 
            \lccode`\~`\!\lowercase{\def~}{\discretionary{\hbox{\char`\!}}{\Wrappedafterbreak}{\hbox{\char`\!}}}% 
            \lccode`\~`\/\lowercase{\def~}{\discretionary{\hbox{\char`\/}}{\Wrappedafterbreak}{\hbox{\char`\/}}}% 
            \catcode`\.\active
            \catcode`\,\active 
            \catcode`\;\active
            \catcode`\:\active
            \catcode`\?\active
            \catcode`\!\active
            \catcode`\/\active 
            \lccode`\~`\~ 	
        }
    \makeatother

    \let\OriginalVerbatim=\Verbatim
    \makeatletter
    \renewcommand{\Verbatim}[1][1]{%
        %\parskip\z@skip
        \sbox\Wrappedcontinuationbox {\Wrappedcontinuationsymbol}%
        \sbox\Wrappedvisiblespacebox {\FV@SetupFont\Wrappedvisiblespace}%
        \def\FancyVerbFormatLine ##1{\hsize\linewidth
            \vtop{\raggedright\hyphenpenalty\z@\exhyphenpenalty\z@
                \doublehyphendemerits\z@\finalhyphendemerits\z@
                \strut ##1\strut}%
        }%
        % If the linebreak is at a space, the latter will be displayed as visible
        % space at end of first line, and a continuation symbol starts next line.
        % Stretch/shrink are however usually zero for typewriter font.
        \def\FV@Space {%
            \nobreak\hskip\z@ plus\fontdimen3\font minus\fontdimen4\font
            \discretionary{\copy\Wrappedvisiblespacebox}{\Wrappedafterbreak}
            {\kern\fontdimen2\font}%
        }%
        
        % Allow breaks at special characters using \PYG... macros.
        \Wrappedbreaksatspecials
        % Breaks at punctuation characters . , ; ? ! and / need catcode=\active 	
        \OriginalVerbatim[#1,codes*=\Wrappedbreaksatpunct]%
    }
    \makeatother

    % Exact colors from NB
    \definecolor{incolor}{HTML}{303F9F}
    \definecolor{outcolor}{HTML}{D84315}
    \definecolor{cellborder}{HTML}{CFCFCF}
    \definecolor{cellbackground}{HTML}{F7F7F7}
    
    % prompt
    \newcommand{\prompt}[4]{
        \llap{{\color{#2}[#3]: #4}}\vspace{-1.25em}
    }
    

    
    % Prevent overflowing lines due to hard-to-break entities
    \sloppy 
    % Setup hyperref package
    \hypersetup{
      breaklinks=true,  % so long urls are correctly broken across lines
      colorlinks=true,
      urlcolor=urlcolor,
      linkcolor=linkcolor,
      citecolor=citecolor,
      }
    % Slightly bigger margins than the latex defaults
    
    \geometry{verbose,tmargin=1in,bmargin=1in,lmargin=1in,rmargin=1in}
    
    

    \begin{document}
    
    
    \maketitle
    
    

    
    Комментарий наставника

Привет! Поздравляю тебя с первым твоим проектом и спасибо, что сдала
задание:) Ты проделала большую работу. Далее в файле мои комментарии ты
сможешь найти в ячейках, аналогичных данфной ( если рамки комментария
зелёные - всё сделано правильно; жёлтые - есть замечания, но не
критично; красные - нужно переделать). Не удаляй эти комментарии и
постарайся учесть их в ходе выполнения проекта.

    \hypertarget{ux438ux441ux441ux43bux435ux434ux43eux432ux430ux43dux438ux435-ux43dux430ux434ux451ux436ux43dux43eux441ux442ux438-ux437ux430ux451ux43cux449ux438ux43aux43eux432}{%
\section{Исследование надёжности
заёмщиков}\label{ux438ux441ux441ux43bux435ux434ux43eux432ux430ux43dux438ux435-ux43dux430ux434ux451ux436ux43dux43eux441ux442ux438-ux437ux430ux451ux43cux449ux438ux43aux43eux432}}

Заказчик --- кредитный отдел банка. Нужно разобраться, влияет ли
семейное положение и количество детей клиента на факт погашения кредита
в срок. Входные данные от банка --- статистика о платёжеспособности
клиентов.

Результаты исследования будут учтены при построении модели
\textbf{кредитного скоринга} --- специальной системы, которая оценивает
способность потенциального заёмщика вернуть кредит банку.

    \hypertarget{ux448ux430ux433-1.-ux43eux442ux43aux440ux43eux439ux442ux435-ux444ux430ux439ux43b-ux441-ux434ux430ux43dux43dux44bux43cux438-ux438-ux438ux437ux443ux447ux438ux442ux435-ux43eux431ux449ux443ux44e-ux438ux43dux444ux43eux440ux43cux430ux446ux438ux44e.}{%
\subsubsection{Шаг 1. Откройте файл с данными и изучите общую
информацию.}\label{ux448ux430ux433-1.-ux43eux442ux43aux440ux43eux439ux442ux435-ux444ux430ux439ux43b-ux441-ux434ux430ux43dux43dux44bux43cux438-ux438-ux438ux437ux443ux447ux438ux442ux435-ux43eux431ux449ux443ux44e-ux438ux43dux444ux43eux440ux43cux430ux446ux438ux44e.}}

    Изучим общую информацию с данными:

    \begin{tcolorbox}[breakable, size=fbox, boxrule=1pt, pad at break*=1mm,colback=cellbackground, colframe=cellborder]
\prompt{In}{incolor}{1}{\hspace{4pt}}
\begin{Verbatim}[commandchars=\\\{\}]
\PY{k+kn}{import} \PY{n+nn}{pandas} \PY{k}{as} \PY{n+nn}{pd}
\PY{n}{data} \PY{o}{=} \PY{n}{pd}\PY{o}{.}\PY{n}{read\PYZus{}csv}\PY{p}{(}\PY{l+s+s1}{\PYZsq{}}\PY{l+s+s1}{/datasets/data.csv}\PY{l+s+s1}{\PYZsq{}}\PY{p}{)}
\PY{n+nb}{print}\PY{p}{(}\PY{n}{data}\PY{o}{.}\PY{n}{head}\PY{p}{(}\PY{l+m+mi}{10}\PY{p}{)}\PY{p}{)}
\PY{c+c1}{\PYZsh{}посмотрим первые 10 строчек таблицы}
\end{Verbatim}
\end{tcolorbox}

    \begin{Verbatim}[commandchars=\\\{\}]
   children  days\_employed  dob\_years education  education\_id  \textbackslash{}
0         1   -8437.673028         42    высшее             0
1         1   -4024.803754         36   среднее             1
2         0   -5623.422610         33   Среднее             1
3         3   -4124.747207         32   среднее             1
4         0  340266.072047         53   среднее             1
5         0    -926.185831         27    высшее             0
6         0   -2879.202052         43    высшее             0
7         0    -152.779569         50   СРЕДНЕЕ             1
8         2   -6929.865299         35    ВЫСШЕЕ             0
9         0   -2188.756445         41   среднее             1

      family\_status  family\_status\_id gender income\_type  debt   total\_income  \textbackslash{}
0   женат / замужем                 0      F   сотрудник     0  253875.639453
1   женат / замужем                 0      F   сотрудник     0  112080.014102
2   женат / замужем                 0      M   сотрудник     0  145885.952297
3   женат / замужем                 0      M   сотрудник     0  267628.550329
4  гражданский брак                 1      F   пенсионер     0  158616.077870
5  гражданский брак                 1      M   компаньон     0  255763.565419
6   женат / замужем                 0      F   компаньон     0  240525.971920
7   женат / замужем                 0      M   сотрудник     0  135823.934197
8  гражданский брак                 1      F   сотрудник     0   95856.832424
9   женат / замужем                 0      M   сотрудник     0  144425.938277

                      purpose
0               покупка жилья
1     приобретение автомобиля
2               покупка жилья
3  дополнительное образование
4             сыграть свадьбу
5               покупка жилья
6           операции с жильем
7                 образование
8       на проведение свадьбы
9     покупка жилья для семьи
\end{Verbatim}

    \begin{tcolorbox}[breakable, size=fbox, boxrule=1pt, pad at break*=1mm,colback=cellbackground, colframe=cellborder]
\prompt{In}{incolor}{2}{\hspace{4pt}}
\begin{Verbatim}[commandchars=\\\{\}]
\PY{n+nb}{print}\PY{p}{(}\PY{n}{data}\PY{o}{.}\PY{n}{tail}\PY{p}{(}\PY{l+m+mi}{10}\PY{p}{)}\PY{p}{)}
\PY{c+c1}{\PYZsh{} посмотрим последние 10 строчек таблицы}
\end{Verbatim}
\end{tcolorbox}

    \begin{Verbatim}[commandchars=\\\{\}]
       children  days\_employed  dob\_years       education  education\_id  \textbackslash{}
21515         1    -467.685130         28         среднее             1
21516         0    -914.391429         42          высшее             0
21517         0    -404.679034         42          высшее             0
21518         0  373995.710838         59         СРЕДНЕЕ             1
21519         1   -2351.431934         37  ученая степень             4
21520         1   -4529.316663         43         среднее             1
21521         0  343937.404131         67         среднее             1
21522         1   -2113.346888         38         среднее             1
21523         3   -3112.481705         38         среднее             1
21524         2   -1984.507589         40         среднее             1

          family\_status  family\_status\_id gender income\_type  debt  \textbackslash{}
21515   женат / замужем                 0      F   сотрудник     1
21516   женат / замужем                 0      F   компаньон     0
21517  гражданский брак                 1      F   компаньон     0
21518   женат / замужем                 0      F   пенсионер     0
21519         в разводе                 3      M   сотрудник     0
21520  гражданский брак                 1      F   компаньон     0
21521   женат / замужем                 0      F   пенсионер     0
21522  гражданский брак                 1      M   сотрудник     1
21523   женат / замужем                 0      M   сотрудник     1
21524   женат / замужем                 0      F   сотрудник     0

        total\_income                            purpose
21515  109486.327999              заняться образованием
21516  322807.776603               покупка своего жилья
21517  178059.553491       на покупку своего автомобиля
21518  153864.650328               сделка с автомобилем
21519  115949.039788  покупка коммерческой недвижимости
21520  224791.862382                  операции с жильем
21521  155999.806512               сделка с автомобилем
21522   89672.561153                       недвижимость
21523  244093.050500       на покупку своего автомобиля
21524   82047.418899              на покупку автомобиля
\end{Verbatim}

    \begin{tcolorbox}[breakable, size=fbox, boxrule=1pt, pad at break*=1mm,colback=cellbackground, colframe=cellborder]
\prompt{In}{incolor}{3}{\hspace{4pt}}
\begin{Verbatim}[commandchars=\\\{\}]
\PY{n}{data}\PY{o}{.}\PY{n}{info}\PY{p}{(}\PY{p}{)}
\end{Verbatim}
\end{tcolorbox}

    \begin{Verbatim}[commandchars=\\\{\}]
<class 'pandas.core.frame.DataFrame'>
RangeIndex: 21525 entries, 0 to 21524
Data columns (total 12 columns):
children            21525 non-null int64
days\_employed       19351 non-null float64
dob\_years           21525 non-null int64
education           21525 non-null object
education\_id        21525 non-null int64
family\_status       21525 non-null object
family\_status\_id    21525 non-null int64
gender              21525 non-null object
income\_type         21525 non-null object
debt                21525 non-null int64
total\_income        19351 non-null float64
purpose             21525 non-null object
dtypes: float64(2), int64(5), object(5)
memory usage: 2.0+ MB
\end{Verbatim}

    \hypertarget{ux432ux44bux432ux43eux434ux44b}{%
\subsubsection{Выводы:}\label{ux432ux44bux432ux43eux434ux44b}}

Наблюдаем следующие возможные сложности в дальнейшем анализе:

\begin{itemize}
\tightlist
\item
  отрицательные значения в столбцах children и days\_employed
\item
  наличие NAN в столбцах days\_employed и total\_income
\item
  прыгающий регистр букв в столбце education
\item
  есть дублирующие признаки
\item
  непонятна валюта месячного дохода
\end{itemize}

    Комментарий наставника

Хорошее начало, так же, для большего понимания данных можно было бы
использовать метод \texttt{describe}

    \hypertarget{ux448ux430ux433-2.-ux43fux440ux435ux434ux43eux431ux440ux430ux431ux43eux442ux43aux430-ux434ux430ux43dux43dux44bux445}{%
\subsubsection{Шаг 2. Предобработка
данных}\label{ux448ux430ux433-2.-ux43fux440ux435ux434ux43eux431ux440ux430ux431ux43eux442ux43aux430-ux434ux430ux43dux43dux44bux445}}

    \hypertarget{ux43eux431ux440ux430ux431ux43eux442ux43aux430-ux43fux440ux43eux43fux443ux441ux43aux43eux432}{%
\subsubsection{Обработка
пропусков}\label{ux43eux431ux440ux430ux431ux43eux442ux43aux430-ux43fux440ux43eux43fux443ux441ux43aux43eux432}}

    \begin{tcolorbox}[breakable, size=fbox, boxrule=1pt, pad at break*=1mm,colback=cellbackground, colframe=cellborder]
\prompt{In}{incolor}{4}{\hspace{4pt}}
\begin{Verbatim}[commandchars=\\\{\}]
\PY{c+c1}{\PYZsh{}посмотрим на суммарное количество пропусков}
\PY{n}{data}\PY{o}{.}\PY{n}{isnull}\PY{p}{(}\PY{p}{)}\PY{o}{.}\PY{n}{sum}\PY{p}{(}\PY{p}{)}
\end{Verbatim}
\end{tcolorbox}

            \begin{tcolorbox}[breakable, boxrule=.5pt, size=fbox, pad at break*=1mm, opacityfill=0]
\prompt{Out}{outcolor}{4}{\hspace{3.5pt}}
\begin{Verbatim}[commandchars=\\\{\}]
children               0
days\_employed       2174
dob\_years              0
education              0
education\_id           0
family\_status          0
family\_status\_id       0
gender                 0
income\_type            0
debt                   0
total\_income        2174
purpose                0
dtype: int64
\end{Verbatim}
\end{tcolorbox}
        
    \begin{tcolorbox}[breakable, size=fbox, boxrule=1pt, pad at break*=1mm,colback=cellbackground, colframe=cellborder]
\prompt{In}{incolor}{5}{\hspace{4pt}}
\begin{Verbatim}[commandchars=\\\{\}]
\PY{c+c1}{\PYZsh{}заменим пропуски в столбце total\PYZus{}income, }
\PY{c+c1}{\PYZsh{}так как в дальнейшем он понадобится для анализа данных }
\PY{c+c1}{\PYZsh{}по запрошенным задачам}
\PY{n}{data}\PY{p}{[}\PY{l+s+s1}{\PYZsq{}}\PY{l+s+s1}{total\PYZus{}income}\PY{l+s+s1}{\PYZsq{}}\PY{p}{]} \PY{o}{=} \PY{n}{data}\PY{p}{[}\PY{l+s+s1}{\PYZsq{}}\PY{l+s+s1}{total\PYZus{}income}\PY{l+s+s1}{\PYZsq{}}\PY{p}{]}\PY{o}{.}\PY{n}{fillna}\PY{p}{(}\PY{l+m+mi}{0}\PY{p}{)}
\end{Verbatim}
\end{tcolorbox}

    \begin{tcolorbox}[breakable, size=fbox, boxrule=1pt, pad at break*=1mm,colback=cellbackground, colframe=cellborder]
\prompt{In}{incolor}{6}{\hspace{4pt}}
\begin{Verbatim}[commandchars=\\\{\}]
\PY{n}{data}\PY{p}{[}\PY{l+s+s1}{\PYZsq{}}\PY{l+s+s1}{days\PYZus{}employed}\PY{l+s+s1}{\PYZsq{}}\PY{p}{]} \PY{o}{=} \PY{n}{data}\PY{p}{[}\PY{l+s+s1}{\PYZsq{}}\PY{l+s+s1}{days\PYZus{}employed}\PY{l+s+s1}{\PYZsq{}}\PY{p}{]}\PY{o}{.}\PY{n}{fillna}\PY{p}{(}\PY{l+m+mi}{0}\PY{p}{)}
\end{Verbatim}
\end{tcolorbox}

    \begin{tcolorbox}[breakable, size=fbox, boxrule=1pt, pad at break*=1mm,colback=cellbackground, colframe=cellborder]
\prompt{In}{incolor}{7}{\hspace{4pt}}
\begin{Verbatim}[commandchars=\\\{\}]
\PY{c+c1}{\PYZsh{}проверяем, остались ли пропуска после замены}
\PY{n}{data}\PY{o}{.}\PY{n}{isnull}\PY{p}{(}\PY{p}{)}\PY{o}{.}\PY{n}{sum}\PY{p}{(}\PY{p}{)}
\end{Verbatim}
\end{tcolorbox}

            \begin{tcolorbox}[breakable, boxrule=.5pt, size=fbox, pad at break*=1mm, opacityfill=0]
\prompt{Out}{outcolor}{7}{\hspace{3.5pt}}
\begin{Verbatim}[commandchars=\\\{\}]
children            0
days\_employed       0
dob\_years           0
education           0
education\_id        0
family\_status       0
family\_status\_id    0
gender              0
income\_type         0
debt                0
total\_income        0
purpose             0
dtype: int64
\end{Verbatim}
\end{tcolorbox}
        
    \hypertarget{ux432ux44bux432ux43eux434}{%
\section{Вывод}\label{ux432ux44bux432ux43eux434}}

\begin{itemize}
\item
  какие пропущенные значения обнаружили: Пропущенные значения обнаружены
  в двух столбцах: days\_employed и total\_income (в 2174 строках из
  21525 строках пропущены значения).
\item
  возможные причины появления пропусков в данных:
\end{itemize}

\begin{enumerate}
\def\labelenumi{\arabic{enumi}.}
\tightlist
\item
  люди намерено пропустили эти пункты, потому что работают не
  официально;
\item
  еще просто не работали
\item
  баг, данные некоректно введены
\end{enumerate}

\begin{itemize}
\tightlist
\item
  объясните, по какому принципу заполнены пропуски: Был применен метод,
  который подразумевает заполнение NaN на 0, применять в данном случае
  метод mean() или mead() было бы некорректным, это могло бы исказить
  данные, также как и удаление пропусков.
\end{itemize}

    Далее будем рассматривать в анализе столбец total\_income, так как
именно он понадобится для решение поставленных клиентом задач.

    Комментарий наставника

Тут лучше не заполнять пропуски просто медианой, средним или нулём,
требуется более обоснованное заполнение пропусков исходя из имеющихся
данных

    \hypertarget{ux437ux430ux43cux435ux43dux430-ux442ux438ux43fux430-ux434ux430ux43dux43dux44bux445}{%
\subsubsection{Замена типа
данных}\label{ux437ux430ux43cux435ux43dux430-ux442ux438ux43fux430-ux434ux430ux43dux43dux44bux445}}

    В запросе заказчика есть следующие вопросы: 1. Есть ли зависимость между
наличием детей и возвратом кредита в срок? 2. Есть ли зависимость между
семейным положением и возвратом кредита в срок? 3. Есть ли зависимость
между уровнем дохода и возвратом кредита в срок? 4. Как разные цели
кредита влияют на его возврат в срок?

В связи с этим, далее будем рассматривать и перерабатывать данные только
в необходимых для анализа столбцах. Если потребуется дополнительная
информация для анализа из оставшихся столбцах, тогда заменим типы данных
и там =)

    \begin{tcolorbox}[breakable, size=fbox, boxrule=1pt, pad at break*=1mm,colback=cellbackground, colframe=cellborder]
\prompt{In}{incolor}{8}{\hspace{4pt}}
\begin{Verbatim}[commandchars=\\\{\}]
\PY{c+c1}{\PYZsh{}посмотрим столбец \PYZdq{}children\PYZdq{}, так как есть запрос: }
\PY{c+c1}{\PYZsh{}влияет ли наличие детей не возврат кредита в срок.}
\PY{n}{data}\PY{p}{[}\PY{l+s+s1}{\PYZsq{}}\PY{l+s+s1}{children}\PY{l+s+s1}{\PYZsq{}}\PY{p}{]}\PY{o}{.}\PY{n}{unique}\PY{p}{(}\PY{p}{)}
\end{Verbatim}
\end{tcolorbox}

            \begin{tcolorbox}[breakable, boxrule=.5pt, size=fbox, pad at break*=1mm, opacityfill=0]
\prompt{Out}{outcolor}{8}{\hspace{3.5pt}}
\begin{Verbatim}[commandchars=\\\{\}]
array([ 1,  0,  3,  2, -1,  4, 20,  5])
\end{Verbatim}
\end{tcolorbox}
        В полученных данных смущают значения -1 и 20. -1 - скорее всего баг и ошибка, а наличие 20 детей предположить можно. Подсчитаем количество строк, где высвечивается значение -1. 
    \begin{tcolorbox}[breakable, size=fbox, boxrule=1pt, pad at break*=1mm,colback=cellbackground, colframe=cellborder]
\prompt{In}{incolor}{9}{\hspace{4pt}}
\begin{Verbatim}[commandchars=\\\{\}]
\PY{n}{data}\PY{p}{[}\PY{n}{data}\PY{p}{[}\PY{l+s+s1}{\PYZsq{}}\PY{l+s+s1}{children}\PY{l+s+s1}{\PYZsq{}}\PY{p}{]} \PY{o}{==} \PY{o}{\PYZhy{}}\PY{l+m+mi}{1}\PY{p}{]}\PY{p}{[}\PY{l+s+s1}{\PYZsq{}}\PY{l+s+s1}{children}\PY{l+s+s1}{\PYZsq{}}\PY{p}{]}\PY{o}{.}\PY{n}{count}\PY{p}{(}\PY{p}{)}
\end{Verbatim}
\end{tcolorbox}

            \begin{tcolorbox}[breakable, boxrule=.5pt, size=fbox, pad at break*=1mm, opacityfill=0]
\prompt{Out}{outcolor}{9}{\hspace{3.5pt}}
\begin{Verbatim}[commandchars=\\\{\}]
47
\end{Verbatim}
\end{tcolorbox}
        
    \begin{tcolorbox}[breakable, size=fbox, boxrule=1pt, pad at break*=1mm,colback=cellbackground, colframe=cellborder]
\prompt{In}{incolor}{10}{\hspace{4pt}}
\begin{Verbatim}[commandchars=\\\{\}]
\PY{c+c1}{\PYZsh{}\PYZhy{}1 можно попробовать заменить на среднее количество детей,}
\PY{c+c1}{\PYZsh{}так как не факт, что имелось ввиду количество одного ребенка. }
\PY{n}{mean\PYZus{}child} \PY{o}{=} \PY{n}{data}\PY{p}{[}\PY{l+s+s1}{\PYZsq{}}\PY{l+s+s1}{children}\PY{l+s+s1}{\PYZsq{}}\PY{p}{]}\PY{o}{.}\PY{n}{mean}\PY{p}{(}\PY{p}{)}
\PY{n+nb}{print}\PY{p}{(}\PY{n}{mean\PYZus{}child}\PY{p}{)}
\end{Verbatim}
\end{tcolorbox}

    \begin{Verbatim}[commandchars=\\\{\}]
0.5389082462253194
\end{Verbatim}

    \begin{tcolorbox}[breakable, size=fbox, boxrule=1pt, pad at break*=1mm,colback=cellbackground, colframe=cellborder]
\prompt{In}{incolor}{11}{\hspace{4pt}}
\begin{Verbatim}[commandchars=\\\{\}]
\PY{c+c1}{\PYZsh{}переведем в целое число для удобства}
\PY{n}{mean\PYZus{}child1} \PY{o}{=} \PY{n+nb}{int}\PY{p}{(}\PY{n}{mean\PYZus{}child}\PY{o}{*}\PY{l+m+mi}{1000000}\PY{p}{)}
\PY{n+nb}{print}\PY{p}{(}\PY{n+nb}{type}\PY{p}{(}\PY{n}{mean\PYZus{}child1}\PY{p}{)}\PY{p}{)}
\end{Verbatim}
\end{tcolorbox}

    \begin{Verbatim}[commandchars=\\\{\}]
<class 'int'>
\end{Verbatim}

    \begin{tcolorbox}[breakable, size=fbox, boxrule=1pt, pad at break*=1mm,colback=cellbackground, colframe=cellborder]
\prompt{In}{incolor}{12}{\hspace{4pt}}
\begin{Verbatim}[commandchars=\\\{\}]
\PY{c+c1}{\PYZsh{}Заменим \PYZhy{}1 на значение mean\PYZus{}child1}
\PY{n}{data}\PY{p}{[}\PY{l+s+s1}{\PYZsq{}}\PY{l+s+s1}{children}\PY{l+s+s1}{\PYZsq{}}\PY{p}{]} \PY{o}{=} \PY{n}{data}\PY{p}{[}\PY{l+s+s1}{\PYZsq{}}\PY{l+s+s1}{children}\PY{l+s+s1}{\PYZsq{}}\PY{p}{]}\PY{o}{.}\PY{n}{replace}\PY{p}{(}\PY{o}{\PYZhy{}}\PY{l+m+mi}{1}\PY{p}{,} \PY{n}{mean\PYZus{}child1}\PY{p}{)}
\end{Verbatim}
\end{tcolorbox}

    \begin{tcolorbox}[breakable, size=fbox, boxrule=1pt, pad at break*=1mm,colback=cellbackground, colframe=cellborder]
\prompt{In}{incolor}{13}{\hspace{4pt}}
\begin{Verbatim}[commandchars=\\\{\}]
\PY{n}{data}\PY{p}{[}\PY{n}{data}\PY{p}{[}\PY{l+s+s1}{\PYZsq{}}\PY{l+s+s1}{children}\PY{l+s+s1}{\PYZsq{}}\PY{p}{]} \PY{o}{==} \PY{o}{\PYZhy{}}\PY{l+m+mi}{1}\PY{p}{]}\PY{p}{[}\PY{l+s+s1}{\PYZsq{}}\PY{l+s+s1}{children}\PY{l+s+s1}{\PYZsq{}}\PY{p}{]}\PY{o}{.}\PY{n}{count}\PY{p}{(}\PY{p}{)}
\end{Verbatim}
\end{tcolorbox}

            \begin{tcolorbox}[breakable, boxrule=.5pt, size=fbox, pad at break*=1mm, opacityfill=0]
\prompt{Out}{outcolor}{13}{\hspace{3.5pt}}
\begin{Verbatim}[commandchars=\\\{\}]
0
\end{Verbatim}
\end{tcolorbox}
        
    \begin{tcolorbox}[breakable, size=fbox, boxrule=1pt, pad at break*=1mm,colback=cellbackground, colframe=cellborder]
\prompt{In}{incolor}{42}{\hspace{4pt}}
\begin{Verbatim}[commandchars=\\\{\}]
\PY{c+c1}{\PYZsh{}Посмотрим сколько строк, где количество детей = 20}
\PY{n}{data}\PY{p}{[}\PY{n}{data}\PY{p}{[}\PY{l+s+s1}{\PYZsq{}}\PY{l+s+s1}{children}\PY{l+s+s1}{\PYZsq{}}\PY{p}{]} \PY{o}{==} \PY{l+m+mi}{20}\PY{p}{]}\PY{p}{[}\PY{l+s+s1}{\PYZsq{}}\PY{l+s+s1}{children}\PY{l+s+s1}{\PYZsq{}}\PY{p}{]}\PY{o}{.}\PY{n}{count}\PY{p}{(}\PY{p}{)}
\end{Verbatim}
\end{tcolorbox}

            \begin{tcolorbox}[breakable, boxrule=.5pt, size=fbox, pad at break*=1mm, opacityfill=0]
\prompt{Out}{outcolor}{42}{\hspace{3.5pt}}
\begin{Verbatim}[commandchars=\\\{\}]
76
\end{Verbatim}
\end{tcolorbox}
        
    \begin{tcolorbox}[breakable, size=fbox, boxrule=1pt, pad at break*=1mm,colback=cellbackground, colframe=cellborder]
\prompt{In}{incolor}{43}{\hspace{4pt}}
\begin{Verbatim}[commandchars=\\\{\}]
\PY{n}{data}\PY{p}{[}\PY{l+s+s1}{\PYZsq{}}\PY{l+s+s1}{children}\PY{l+s+s1}{\PYZsq{}}\PY{p}{]}\PY{o}{.}\PY{n}{astype}\PY{p}{(}\PY{l+s+s1}{\PYZsq{}}\PY{l+s+s1}{int}\PY{l+s+s1}{\PYZsq{}}\PY{p}{)}
\end{Verbatim}
\end{tcolorbox}

            \begin{tcolorbox}[breakable, boxrule=.5pt, size=fbox, pad at break*=1mm, opacityfill=0]
\prompt{Out}{outcolor}{43}{\hspace{3.5pt}}
\begin{Verbatim}[commandchars=\\\{\}]
0        1
1        1
2        0
3        3
4        0
        ..
21466    1
21467    0
21468    1
21469    3
21470    2
Name: children, Length: 21471, dtype: int64
\end{Verbatim}
\end{tcolorbox}
        
    В постановке задачи не стоит вопрос о количестве детей, стоит вопрос
влияет ли наличие детей на возврат кредита в срок? Соотвтетсвенно, нас
интересует ответ на вопрос есть дети у клиента или нет. Далее в проекте
распределим эти данные на категории.

    \begin{tcolorbox}[breakable, size=fbox, boxrule=1pt, pad at break*=1mm,colback=cellbackground, colframe=cellborder]
\prompt{In}{incolor}{16}{\hspace{4pt}}
\begin{Verbatim}[commandchars=\\\{\}]
\PY{c+c1}{\PYZsh{}теперь посмотрим столбец \PYZdq{}family\PYZus{}status\PYZdq{} для дальнейшего анализа }
\PY{n}{data}\PY{p}{[}\PY{l+s+s1}{\PYZsq{}}\PY{l+s+s1}{family\PYZus{}status}\PY{l+s+s1}{\PYZsq{}}\PY{p}{]}\PY{o}{.}\PY{n}{unique}\PY{p}{(}\PY{p}{)}
\end{Verbatim}
\end{tcolorbox}

            \begin{tcolorbox}[breakable, boxrule=.5pt, size=fbox, pad at break*=1mm, opacityfill=0]
\prompt{Out}{outcolor}{16}{\hspace{3.5pt}}
\begin{Verbatim}[commandchars=\\\{\}]
array(['женат / замужем', 'гражданский брак', 'вдовец / вдова',
       'в разводе', 'Не женат / не замужем'], dtype=object)
\end{Verbatim}
\end{tcolorbox}
        
    \begin{tcolorbox}[breakable, size=fbox, boxrule=1pt, pad at break*=1mm,colback=cellbackground, colframe=cellborder]
\prompt{In}{incolor}{17}{\hspace{4pt}}
\begin{Verbatim}[commandchars=\\\{\}]
\PY{c+c1}{\PYZsh{}приведем все значения к нижнему регистру}
\PY{n}{data}\PY{p}{[}\PY{l+s+s1}{\PYZsq{}}\PY{l+s+s1}{family\PYZus{}status}\PY{l+s+s1}{\PYZsq{}}\PY{p}{]} \PY{o}{=} \PY{n}{data}\PY{p}{[}\PY{l+s+s1}{\PYZsq{}}\PY{l+s+s1}{family\PYZus{}status}\PY{l+s+s1}{\PYZsq{}}\PY{p}{]}\PY{o}{.}\PY{n}{str}\PY{o}{.}\PY{n}{lower}\PY{p}{(}\PY{p}{)}
\PY{n}{data}\PY{p}{[}\PY{l+s+s1}{\PYZsq{}}\PY{l+s+s1}{family\PYZus{}status}\PY{l+s+s1}{\PYZsq{}}\PY{p}{]}\PY{o}{.}\PY{n}{unique}\PY{p}{(}\PY{p}{)}
\end{Verbatim}
\end{tcolorbox}

            \begin{tcolorbox}[breakable, boxrule=.5pt, size=fbox, pad at break*=1mm, opacityfill=0]
\prompt{Out}{outcolor}{17}{\hspace{3.5pt}}
\begin{Verbatim}[commandchars=\\\{\}]
array(['женат / замужем', 'гражданский брак', 'вдовец / вдова',
       'в разводе', 'не женат / не замужем'], dtype=object)
\end{Verbatim}
\end{tcolorbox}
        
    \begin{tcolorbox}[breakable, size=fbox, boxrule=1pt, pad at break*=1mm,colback=cellbackground, colframe=cellborder]
\prompt{In}{incolor}{18}{\hspace{4pt}}
\begin{Verbatim}[commandchars=\\\{\}]
\PY{c+c1}{\PYZsh{}рассмотрим уровень дохода}
\PY{n+nb}{print}\PY{p}{(}\PY{n}{data}\PY{p}{[}\PY{l+s+s1}{\PYZsq{}}\PY{l+s+s1}{total\PYZus{}income}\PY{l+s+s1}{\PYZsq{}}\PY{p}{]}\PY{p}{)}
\end{Verbatim}
\end{tcolorbox}

    \begin{Verbatim}[commandchars=\\\{\}]
0        253875.639453
1        112080.014102
2        145885.952297
3        267628.550329
4        158616.077870
             {\ldots}
21520    224791.862382
21521    155999.806512
21522     89672.561153
21523    244093.050500
21524     82047.418899
Name: total\_income, Length: 21525, dtype: float64
\end{Verbatim}

    \begin{tcolorbox}[breakable, size=fbox, boxrule=1pt, pad at break*=1mm,colback=cellbackground, colframe=cellborder]
\prompt{In}{incolor}{19}{\hspace{4pt}}
\begin{Verbatim}[commandchars=\\\{\}]
\PY{c+c1}{\PYZsh{}приведем данные в более удобный для анализа вид}
\PY{n}{data}\PY{p}{[}\PY{l+s+s1}{\PYZsq{}}\PY{l+s+s1}{total\PYZus{}income}\PY{l+s+s1}{\PYZsq{}}\PY{p}{]} \PY{o}{=} \PY{n}{data}\PY{p}{[}\PY{l+s+s1}{\PYZsq{}}\PY{l+s+s1}{total\PYZus{}income}\PY{l+s+s1}{\PYZsq{}}\PY{p}{]}\PY{o}{.}\PY{n}{astype}\PY{p}{(}\PY{n+nb}{int}\PY{p}{)}
\PY{n}{data}\PY{o}{.}\PY{n}{head}\PY{p}{(}\PY{l+m+mi}{10}\PY{p}{)}
\end{Verbatim}
\end{tcolorbox}

            \begin{tcolorbox}[breakable, boxrule=.5pt, size=fbox, pad at break*=1mm, opacityfill=0]
\prompt{Out}{outcolor}{19}{\hspace{3.5pt}}
\begin{Verbatim}[commandchars=\\\{\}]
   children  days\_employed  dob\_years education  education\_id  \textbackslash{}
0         1   -8437.673028         42    высшее             0
1         1   -4024.803754         36   среднее             1
2         0   -5623.422610         33   Среднее             1
3         3   -4124.747207         32   среднее             1
4         0  340266.072047         53   среднее             1
5         0    -926.185831         27    высшее             0
6         0   -2879.202052         43    высшее             0
7         0    -152.779569         50   СРЕДНЕЕ             1
8         2   -6929.865299         35    ВЫСШЕЕ             0
9         0   -2188.756445         41   среднее             1

      family\_status  family\_status\_id gender income\_type  debt  total\_income  \textbackslash{}
0   женат / замужем                 0      F   сотрудник     0        253875
1   женат / замужем                 0      F   сотрудник     0        112080
2   женат / замужем                 0      M   сотрудник     0        145885
3   женат / замужем                 0      M   сотрудник     0        267628
4  гражданский брак                 1      F   пенсионер     0        158616
5  гражданский брак                 1      M   компаньон     0        255763
6   женат / замужем                 0      F   компаньон     0        240525
7   женат / замужем                 0      M   сотрудник     0        135823
8  гражданский брак                 1      F   сотрудник     0         95856
9   женат / замужем                 0      M   сотрудник     0        144425

                      purpose
0               покупка жилья
1     приобретение автомобиля
2               покупка жилья
3  дополнительное образование
4             сыграть свадьбу
5               покупка жилья
6           операции с жильем
7                 образование
8       на проведение свадьбы
9     покупка жилья для семьи
\end{Verbatim}
\end{tcolorbox}
        
    \begin{tcolorbox}[breakable, size=fbox, boxrule=1pt, pad at break*=1mm,colback=cellbackground, colframe=cellborder]
\prompt{In}{incolor}{20}{\hspace{4pt}}
\begin{Verbatim}[commandchars=\\\{\}]
\PY{c+c1}{\PYZsh{}посмотрим на следующий стобец для дальнейшего анализа \PYZdq{}purpose\PYZdq{}}
\PY{n}{data}\PY{p}{[}\PY{l+s+s1}{\PYZsq{}}\PY{l+s+s1}{purpose}\PY{l+s+s1}{\PYZsq{}}\PY{p}{]}\PY{o}{.}\PY{n}{unique}\PY{p}{(}\PY{p}{)}
\end{Verbatim}
\end{tcolorbox}

            \begin{tcolorbox}[breakable, boxrule=.5pt, size=fbox, pad at break*=1mm, opacityfill=0]
\prompt{Out}{outcolor}{20}{\hspace{3.5pt}}
\begin{Verbatim}[commandchars=\\\{\}]
array(['покупка жилья', 'приобретение автомобиля',
       'дополнительное образование', 'сыграть свадьбу',
       'операции с жильем', 'образование', 'на проведение свадьбы',
       'покупка жилья для семьи', 'покупка недвижимости',
       'покупка коммерческой недвижимости', 'покупка жилой недвижимости',
       'строительство собственной недвижимости', 'недвижимость',
       'строительство недвижимости', 'на покупку подержанного автомобиля',
       'на покупку своего автомобиля',
       'операции с коммерческой недвижимостью',
       'строительство жилой недвижимости', 'жилье',
       'операции со своей недвижимостью', 'автомобили',
       'заняться образованием', 'сделка с подержанным автомобилем',
       'получение образования', 'автомобиль', 'свадьба',
       'получение дополнительного образования', 'покупка своего жилья',
       'операции с недвижимостью', 'получение высшего образования',
       'свой автомобиль', 'сделка с автомобилем',
       'профильное образование', 'высшее образование',
       'покупка жилья для сдачи', 'на покупку автомобиля', 'ремонт жилью',
       'заняться высшим образованием'], dtype=object)
\end{Verbatim}
\end{tcolorbox}
        
    \begin{tcolorbox}[breakable, size=fbox, boxrule=1pt, pad at break*=1mm,colback=cellbackground, colframe=cellborder]
\prompt{In}{incolor}{21}{\hspace{4pt}}
\begin{Verbatim}[commandchars=\\\{\}]
\PY{c+c1}{\PYZsh{}dept}
\PY{n}{data}\PY{p}{[}\PY{l+s+s1}{\PYZsq{}}\PY{l+s+s1}{debt}\PY{l+s+s1}{\PYZsq{}}\PY{p}{]}\PY{o}{.}\PY{n}{value\PYZus{}counts}\PY{p}{(}\PY{p}{)}
\end{Verbatim}
\end{tcolorbox}

            \begin{tcolorbox}[breakable, boxrule=.5pt, size=fbox, pad at break*=1mm, opacityfill=0]
\prompt{Out}{outcolor}{21}{\hspace{3.5pt}}
\begin{Verbatim}[commandchars=\\\{\}]
0    19784
1     1741
Name: debt, dtype: int64
\end{Verbatim}
\end{tcolorbox}
        
    В блоке ``Замена типов данных'' мы рассмотрели все необходимые для
дальнейшего анализа поставленных вопросов заказчиком столбцы. Необходимо
доработать стобец - ``purpose'' (далее будет к нему применен метод
лемматизации),

    Комментарий наставника

Замена типа данных выполнена верно

    \hypertarget{ux43eux431ux440ux430ux431ux43eux442ux43aux430-ux434ux443ux431ux43bux438ux43aux430ux442ux43eux432}{%
\subsubsection{Обработка
дубликатов}\label{ux43eux431ux440ux430ux431ux43eux442ux43aux430-ux434ux443ux431ux43bux438ux43aux430ux442ux43eux432}}

    \begin{tcolorbox}[breakable, size=fbox, boxrule=1pt, pad at break*=1mm,colback=cellbackground, colframe=cellborder]
\prompt{In}{incolor}{22}{\hspace{4pt}}
\begin{Verbatim}[commandchars=\\\{\}]
\PY{c+c1}{\PYZsh{}проверим наличие дубликатов}
\PY{n}{data}\PY{o}{.}\PY{n}{duplicated}\PY{p}{(}\PY{p}{)}\PY{o}{.}\PY{n}{sum}\PY{p}{(}\PY{p}{)}
\end{Verbatim}
\end{tcolorbox}

            \begin{tcolorbox}[breakable, boxrule=.5pt, size=fbox, pad at break*=1mm, opacityfill=0]
\prompt{Out}{outcolor}{22}{\hspace{3.5pt}}
\begin{Verbatim}[commandchars=\\\{\}]
54
\end{Verbatim}
\end{tcolorbox}
        
    Возможо дубли появились при заполнении данных, удалим и проверим заново
дубли.

    \begin{tcolorbox}[breakable, size=fbox, boxrule=1pt, pad at break*=1mm,colback=cellbackground, colframe=cellborder]
\prompt{In}{incolor}{23}{\hspace{4pt}}
\begin{Verbatim}[commandchars=\\\{\}]
\PY{n}{data} \PY{o}{=} \PY{n}{data}\PY{o}{.}\PY{n}{drop\PYZus{}duplicates}\PY{p}{(}\PY{p}{)}\PY{o}{.}\PY{n}{reset\PYZus{}index}\PY{p}{(}\PY{n}{drop}\PY{o}{=}\PY{k+kc}{True}\PY{p}{)}
\end{Verbatim}
\end{tcolorbox}

    \begin{tcolorbox}[breakable, size=fbox, boxrule=1pt, pad at break*=1mm,colback=cellbackground, colframe=cellborder]
\prompt{In}{incolor}{24}{\hspace{4pt}}
\begin{Verbatim}[commandchars=\\\{\}]
\PY{n}{data}\PY{o}{.}\PY{n}{duplicated}\PY{p}{(}\PY{p}{)}\PY{o}{.}\PY{n}{sum}\PY{p}{(}\PY{p}{)}
\end{Verbatim}
\end{tcolorbox}

            \begin{tcolorbox}[breakable, boxrule=.5pt, size=fbox, pad at break*=1mm, opacityfill=0]
\prompt{Out}{outcolor}{24}{\hspace{3.5pt}}
\begin{Verbatim}[commandchars=\\\{\}]
0
\end{Verbatim}
\end{tcolorbox}
        
    \begin{tcolorbox}[breakable, size=fbox, boxrule=1pt, pad at break*=1mm,colback=cellbackground, colframe=cellborder]
\prompt{In}{incolor}{25}{\hspace{4pt}}
\begin{Verbatim}[commandchars=\\\{\}]
\PY{n}{data}\PY{o}{.}\PY{n}{head}\PY{p}{(}\PY{l+m+mi}{10}\PY{p}{)}
\end{Verbatim}
\end{tcolorbox}

            \begin{tcolorbox}[breakable, boxrule=.5pt, size=fbox, pad at break*=1mm, opacityfill=0]
\prompt{Out}{outcolor}{25}{\hspace{3.5pt}}
\begin{Verbatim}[commandchars=\\\{\}]
   children  days\_employed  dob\_years education  education\_id  \textbackslash{}
0         1   -8437.673028         42    высшее             0
1         1   -4024.803754         36   среднее             1
2         0   -5623.422610         33   Среднее             1
3         3   -4124.747207         32   среднее             1
4         0  340266.072047         53   среднее             1
5         0    -926.185831         27    высшее             0
6         0   -2879.202052         43    высшее             0
7         0    -152.779569         50   СРЕДНЕЕ             1
8         2   -6929.865299         35    ВЫСШЕЕ             0
9         0   -2188.756445         41   среднее             1

      family\_status  family\_status\_id gender income\_type  debt  total\_income  \textbackslash{}
0   женат / замужем                 0      F   сотрудник     0        253875
1   женат / замужем                 0      F   сотрудник     0        112080
2   женат / замужем                 0      M   сотрудник     0        145885
3   женат / замужем                 0      M   сотрудник     0        267628
4  гражданский брак                 1      F   пенсионер     0        158616
5  гражданский брак                 1      M   компаньон     0        255763
6   женат / замужем                 0      F   компаньон     0        240525
7   женат / замужем                 0      M   сотрудник     0        135823
8  гражданский брак                 1      F   сотрудник     0         95856
9   женат / замужем                 0      M   сотрудник     0        144425

                      purpose
0               покупка жилья
1     приобретение автомобиля
2               покупка жилья
3  дополнительное образование
4             сыграть свадьбу
5               покупка жилья
6           операции с жильем
7                 образование
8       на проведение свадьбы
9     покупка жилья для семьи
\end{Verbatim}
\end{tcolorbox}
        
    \hypertarget{ux432ux44bux432ux43eux434}{%
\subsubsection{Вывод}\label{ux432ux44bux432ux43eux434}}

Проверили на наличие дубликатов, которые, возможно, возникли из-за
неправильного заполнения данных. Дубликаты удалили и заново проверили их
на наличие. Были применены методы duplicated() и drop\_duplicates().

    Комментарий наставника

Дубликаты найдены и обработаны верно,отлично

    \hypertarget{ux43bux435ux43cux43cux430ux442ux438ux437ux430ux446ux438ux44f}{%
\subsubsection{Лемматизация}\label{ux43bux435ux43cux43cux430ux442ux438ux437ux430ux446ux438ux44f}}

    \begin{tcolorbox}[breakable, size=fbox, boxrule=1pt, pad at break*=1mm,colback=cellbackground, colframe=cellborder]
\prompt{In}{incolor}{26}{\hspace{4pt}}
\begin{Verbatim}[commandchars=\\\{\}]
\PY{c+c1}{\PYZsh{}Для начала выведем еще раз список значений в столбце \PYZdq{}purpose\PYZdq{}}
\PY{n}{data}\PY{p}{[}\PY{l+s+s1}{\PYZsq{}}\PY{l+s+s1}{purpose}\PY{l+s+s1}{\PYZsq{}}\PY{p}{]}\PY{o}{.}\PY{n}{unique}\PY{p}{(}\PY{p}{)}\PY{o}{.}\PY{n}{tolist}\PY{p}{(}\PY{p}{)}
\end{Verbatim}
\end{tcolorbox}

            \begin{tcolorbox}[breakable, boxrule=.5pt, size=fbox, pad at break*=1mm, opacityfill=0]
\prompt{Out}{outcolor}{26}{\hspace{3.5pt}}
\begin{Verbatim}[commandchars=\\\{\}]
['покупка жилья',
 'приобретение автомобиля',
 'дополнительное образование',
 'сыграть свадьбу',
 'операции с жильем',
 'образование',
 'на проведение свадьбы',
 'покупка жилья для семьи',
 'покупка недвижимости',
 'покупка коммерческой недвижимости',
 'покупка жилой недвижимости',
 'строительство собственной недвижимости',
 'недвижимость',
 'строительство недвижимости',
 'на покупку подержанного автомобиля',
 'на покупку своего автомобиля',
 'операции с коммерческой недвижимостью',
 'строительство жилой недвижимости',
 'жилье',
 'операции со своей недвижимостью',
 'автомобили',
 'заняться образованием',
 'сделка с подержанным автомобилем',
 'получение образования',
 'автомобиль',
 'свадьба',
 'получение дополнительного образования',
 'покупка своего жилья',
 'операции с недвижимостью',
 'получение высшего образования',
 'свой автомобиль',
 'сделка с автомобилем',
 'профильное образование',
 'высшее образование',
 'покупка жилья для сдачи',
 'на покупку автомобиля',
 'ремонт жилью',
 'заняться высшим образованием']
\end{Verbatim}
\end{tcolorbox}
        
    \begin{tcolorbox}[breakable, size=fbox, boxrule=1pt, pad at break*=1mm,colback=cellbackground, colframe=cellborder]
\prompt{In}{incolor}{27}{\hspace{4pt}}
\begin{Verbatim}[commandchars=\\\{\}]
\PY{c+c1}{\PYZsh{}Вручную выделим основные цели взятия кредита:}
   \PY{c+c1}{\PYZsh{} \PYZhy{} покупка или ремонт автомобиля}
   \PY{c+c1}{\PYZsh{} \PYZhy{} жилье или недвижимость (как коммерческая так и для собсвенного пользования)}
   \PY{c+c1}{\PYZsh{} \PYZhy{} получение дополнительного образования}
   \PY{c+c1}{\PYZsh{} \PYZhy{} свадьба}
\PY{c+c1}{\PYZsh{}Получается в данной таблице представлено 4 цели на взятие кредита: автомобиль, жилье/недвижимость, образование и свадьба. }
\end{Verbatim}
\end{tcolorbox}

    \begin{tcolorbox}[breakable, size=fbox, boxrule=1pt, pad at break*=1mm,colback=cellbackground, colframe=cellborder]
\prompt{In}{incolor}{28}{\hspace{4pt}}
\begin{Verbatim}[commandchars=\\\{\}]
\PY{k+kn}{from} \PY{n+nn}{pymystem3} \PY{k+kn}{import} \PY{n}{Mystem}
\PY{k+kn}{from} \PY{n+nn}{collections} \PY{k+kn}{import} \PY{n}{Counter}
\PY{n}{m} \PY{o}{=} \PY{n}{Mystem}\PY{p}{(}\PY{p}{)}

\PY{k}{def} \PY{n+nf}{lemmatize\PYZus{}it}\PY{p}{(}\PY{n}{row}\PY{p}{)}\PY{p}{:}
        \PY{n}{lemmas} \PY{o}{=} \PY{n}{m}\PY{o}{.}\PY{n}{lemmatize}\PY{p}{(}\PY{n}{row}\PY{p}{)}
        \PY{k}{if} \PY{l+s+s1}{\PYZsq{}}\PY{l+s+s1}{автомобиль}\PY{l+s+s1}{\PYZsq{}} \PY{o+ow}{in} \PY{n}{lemmas}\PY{p}{:}
            \PY{k}{return} \PY{l+s+s1}{\PYZsq{}}\PY{l+s+s1}{автокредит}\PY{l+s+s1}{\PYZsq{}}
        \PY{k}{if} \PY{l+s+s1}{\PYZsq{}}\PY{l+s+s1}{свадьба}\PY{l+s+s1}{\PYZsq{}} \PY{o+ow}{in} \PY{n}{lemmas}\PY{p}{:}
            \PY{k}{return} \PY{l+s+s1}{\PYZsq{}}\PY{l+s+s1}{кредит на свадьбу}\PY{l+s+s1}{\PYZsq{}}
        \PY{k}{if} \PY{l+s+s1}{\PYZsq{}}\PY{l+s+s1}{жилье}\PY{l+s+s1}{\PYZsq{}} \PY{o+ow}{in} \PY{n}{lemmas} \PY{o+ow}{or} \PY{l+s+s1}{\PYZsq{}}\PY{l+s+s1}{недвижимость}\PY{l+s+s1}{\PYZsq{}} \PY{o+ow}{in} \PY{n}{lemmas}\PY{p}{:}
            \PY{k}{return} \PY{l+s+s1}{\PYZsq{}}\PY{l+s+s1}{ипотека}\PY{l+s+s1}{\PYZsq{}}
        \PY{k}{if} \PY{l+s+s1}{\PYZsq{}}\PY{l+s+s1}{образование}\PY{l+s+s1}{\PYZsq{}} \PY{o+ow}{in} \PY{n}{lemmas}\PY{p}{:}
            \PY{k}{return} \PY{l+s+s1}{\PYZsq{}}\PY{l+s+s1}{кредит на образование}\PY{l+s+s1}{\PYZsq{}}
        
\PY{n}{data}\PY{p}{[}\PY{l+s+s1}{\PYZsq{}}\PY{l+s+s1}{purpose\PYZus{}group}\PY{l+s+s1}{\PYZsq{}}\PY{p}{]} \PY{o}{=} \PY{n}{data}\PY{p}{[}\PY{l+s+s1}{\PYZsq{}}\PY{l+s+s1}{purpose}\PY{l+s+s1}{\PYZsq{}}\PY{p}{]}\PY{o}{.}\PY{n}{apply}\PY{p}{(}\PY{n}{lemmatize\PYZus{}it}\PY{p}{)}
\end{Verbatim}
\end{tcolorbox}

    \begin{tcolorbox}[breakable, size=fbox, boxrule=1pt, pad at break*=1mm,colback=cellbackground, colframe=cellborder]
\prompt{In}{incolor}{29}{\hspace{4pt}}
\begin{Verbatim}[commandchars=\\\{\}]
\PY{c+c1}{\PYZsh{}почему то RUN не срабатывает для принта: print(data[\PYZsq{}purpose\PYZus{}group\PYZsq{}]), }
\PY{c+c1}{\PYZsh{}поэтому пришлось выводить таким образом =)}
\PY{n+nb}{print}\PY{p}{(}\PY{n}{data}\PY{p}{[}\PY{l+s+s1}{\PYZsq{}}\PY{l+s+s1}{purpose}\PY{l+s+s1}{\PYZsq{}}\PY{p}{]}\PY{o}{.}\PY{n}{apply}\PY{p}{(}\PY{n}{lemmatize\PYZus{}it}\PY{p}{)}\PY{p}{)}
\end{Verbatim}
\end{tcolorbox}

    \begin{Verbatim}[commandchars=\\\{\}]
0                      ипотека
1                   автокредит
2                      ипотека
3        кредит на образование
4            кредит на свадьбу
                 {\ldots}
21466                  ипотека
21467               автокредит
21468                  ипотека
21469               автокредит
21470               автокредит
Name: purpose, Length: 21471, dtype: object
\end{Verbatim}

    \begin{tcolorbox}[breakable, size=fbox, boxrule=1pt, pad at break*=1mm,colback=cellbackground, colframe=cellborder]
\prompt{In}{incolor}{30}{\hspace{4pt}}
\begin{Verbatim}[commandchars=\\\{\}]
\PY{n+nb}{print}\PY{p}{(}\PY{n}{data}\PY{p}{[}\PY{l+s+s1}{\PYZsq{}}\PY{l+s+s1}{purpose\PYZus{}group}\PY{l+s+s1}{\PYZsq{}}\PY{p}{]}\PY{p}{)}
\end{Verbatim}
\end{tcolorbox}

    \begin{Verbatim}[commandchars=\\\{\}]
0                      ипотека
1                   автокредит
2                      ипотека
3        кредит на образование
4            кредит на свадьбу
                 {\ldots}
21466                  ипотека
21467               автокредит
21468                  ипотека
21469               автокредит
21470               автокредит
Name: purpose\_group, Length: 21471, dtype: object
\end{Verbatim}

    \begin{tcolorbox}[breakable, size=fbox, boxrule=1pt, pad at break*=1mm,colback=cellbackground, colframe=cellborder]
\prompt{In}{incolor}{31}{\hspace{4pt}}
\begin{Verbatim}[commandchars=\\\{\}]
\PY{n}{purpose\PYZus{}data} \PY{o}{=} \PY{n}{data}\PY{p}{[}\PY{p}{[}\PY{l+s+s1}{\PYZsq{}}\PY{l+s+s1}{purpose}\PY{l+s+s1}{\PYZsq{}}\PY{p}{,} \PY{l+s+s1}{\PYZsq{}}\PY{l+s+s1}{purpose\PYZus{}group}\PY{l+s+s1}{\PYZsq{}}\PY{p}{]}\PY{p}{]}
\PY{n+nb}{print}\PY{p}{(}\PY{n}{purpose\PYZus{}data}\PY{o}{.}\PY{n}{head}\PY{p}{(}\PY{l+m+mi}{10}\PY{p}{)}\PY{p}{)}
\end{Verbatim}
\end{tcolorbox}

    \begin{Verbatim}[commandchars=\\\{\}]
                      purpose          purpose\_group
0               покупка жилья                ипотека
1     приобретение автомобиля             автокредит
2               покупка жилья                ипотека
3  дополнительное образование  кредит на образование
4             сыграть свадьбу      кредит на свадьбу
5               покупка жилья                ипотека
6           операции с жильем                ипотека
7                 образование  кредит на образование
8       на проведение свадьбы      кредит на свадьбу
9     покупка жилья для семьи                ипотека
\end{Verbatim}

    \hypertarget{ux432ux44bux432ux43eux434}{%
\subsubsection{Вывод}\label{ux432ux44bux432ux43eux434}}

    Выделили основные леммы, а именно цели для преоретение кредита: -
ипотека - кредит на образование - кредит на свадьбу - автокредит

    Комментарий наставника

Лемматизация проведена верно

    \hypertarget{ux43aux430ux442ux435ux433ux43eux440ux438ux437ux430ux446ux438ux44f-ux434ux430ux43dux43dux44bux445}{%
\subsubsection{Категоризация
данных}\label{ux43aux430ux442ux435ux433ux43eux440ux438ux437ux430ux446ux438ux44f-ux434ux430ux43dux43dux44bux445}}

    Из необходимых для дальнейшего анализа столбцов, для категоризации
данных можно рассмотреть два: children и total\_income.

    \begin{tcolorbox}[breakable, size=fbox, boxrule=1pt, pad at break*=1mm,colback=cellbackground, colframe=cellborder]
\prompt{In}{incolor}{32}{\hspace{4pt}}
\begin{Verbatim}[commandchars=\\\{\}]
\PY{c+c1}{\PYZsh{}Категоризуем столбец \PYZdq{}total\PYZus{}income\PYZdq{}, }
\PY{c+c1}{\PYZsh{}для начала посмотрим на данные этого столбца}
\PY{n}{data}\PY{p}{[}\PY{l+s+s1}{\PYZsq{}}\PY{l+s+s1}{total\PYZus{}income}\PY{l+s+s1}{\PYZsq{}}\PY{p}{]}\PY{o}{.}\PY{n}{value\PYZus{}counts}\PY{p}{(}\PY{p}{)}
\end{Verbatim}
\end{tcolorbox}

            \begin{tcolorbox}[breakable, boxrule=.5pt, size=fbox, pad at break*=1mm, opacityfill=0]
\prompt{Out}{outcolor}{32}{\hspace{3.5pt}}
\begin{Verbatim}[commandchars=\\\{\}]
0         2120
204827       3
163060       3
144533       3
121931       3
          {\ldots}
109583       1
101387       1
138249       1
280240       1
229304       1
Name: total\_income, Length: 18607, dtype: int64
\end{Verbatim}
\end{tcolorbox}
        
    \begin{tcolorbox}[breakable, size=fbox, boxrule=1pt, pad at break*=1mm,colback=cellbackground, colframe=cellborder]
\prompt{In}{incolor}{33}{\hspace{4pt}}
\begin{Verbatim}[commandchars=\\\{\}]
\PY{c+c1}{\PYZsh{}Из этого списка сложно сделать анализ, }
\PY{c+c1}{\PYZsh{}поэтому необходимо разбить доход на категории по статусу:}
\PY{k}{def} \PY{n+nf}{income\PYZus{}category}\PY{p}{(}\PY{n}{total\PYZus{}income}\PY{p}{)}\PY{p}{:}
    \PY{k}{if} \PY{n}{total\PYZus{}income} \PY{o}{\PYZlt{}}\PY{o}{=} \PY{l+m+mf}{45000.0}\PY{p}{:}
        \PY{k}{return} \PY{l+s+s1}{\PYZsq{}}\PY{l+s+s1}{низкий уровень дохода}\PY{l+s+s1}{\PYZsq{}}
    \PY{k}{if} \PY{n}{total\PYZus{}income} \PY{o}{\PYZlt{}}\PY{o}{=} \PY{l+m+mf}{90000.0}\PY{p}{:}
        \PY{k}{return} \PY{l+s+s1}{\PYZsq{}}\PY{l+s+s1}{средний уровень дохода}\PY{l+s+s1}{\PYZsq{}}
    \PY{k}{if} \PY{n}{total\PYZus{}income} \PY{o}{\PYZlt{}}\PY{o}{=} \PY{l+m+mf}{180000.0}\PY{p}{:} 
        \PY{k}{return} \PY{l+s+s1}{\PYZsq{}}\PY{l+s+s1}{высокий уровень дохода}\PY{l+s+s1}{\PYZsq{}}
    \PY{k}{return} \PY{l+s+s1}{\PYZsq{}}\PY{l+s+s1}{очень высокий уровень дохода}\PY{l+s+s1}{\PYZsq{}}

\PY{n}{data}\PY{p}{[}\PY{l+s+s1}{\PYZsq{}}\PY{l+s+s1}{income\PYZus{}category}\PY{l+s+s1}{\PYZsq{}}\PY{p}{]} \PY{o}{=} \PY{n}{data}\PY{p}{[}\PY{l+s+s1}{\PYZsq{}}\PY{l+s+s1}{total\PYZus{}income}\PY{l+s+s1}{\PYZsq{}}\PY{p}{]}\PY{o}{.}\PY{n}{apply}\PY{p}{(}\PY{n}{income\PYZus{}category}\PY{p}{)}
\end{Verbatim}
\end{tcolorbox}

    \begin{tcolorbox}[breakable, size=fbox, boxrule=1pt, pad at break*=1mm,colback=cellbackground, colframe=cellborder]
\prompt{In}{incolor}{34}{\hspace{4pt}}
\begin{Verbatim}[commandchars=\\\{\}]
\PY{n}{data}\PY{o}{.}\PY{n}{head}\PY{p}{(}\PY{l+m+mi}{10}\PY{p}{)}
\end{Verbatim}
\end{tcolorbox}

            \begin{tcolorbox}[breakable, boxrule=.5pt, size=fbox, pad at break*=1mm, opacityfill=0]
\prompt{Out}{outcolor}{34}{\hspace{3.5pt}}
\begin{Verbatim}[commandchars=\\\{\}]
   children  days\_employed  dob\_years education  education\_id  \textbackslash{}
0         1   -8437.673028         42    высшее             0
1         1   -4024.803754         36   среднее             1
2         0   -5623.422610         33   Среднее             1
3         3   -4124.747207         32   среднее             1
4         0  340266.072047         53   среднее             1
5         0    -926.185831         27    высшее             0
6         0   -2879.202052         43    высшее             0
7         0    -152.779569         50   СРЕДНЕЕ             1
8         2   -6929.865299         35    ВЫСШЕЕ             0
9         0   -2188.756445         41   среднее             1

      family\_status  family\_status\_id gender income\_type  debt  total\_income  \textbackslash{}
0   женат / замужем                 0      F   сотрудник     0        253875
1   женат / замужем                 0      F   сотрудник     0        112080
2   женат / замужем                 0      M   сотрудник     0        145885
3   женат / замужем                 0      M   сотрудник     0        267628
4  гражданский брак                 1      F   пенсионер     0        158616
5  гражданский брак                 1      M   компаньон     0        255763
6   женат / замужем                 0      F   компаньон     0        240525
7   женат / замужем                 0      M   сотрудник     0        135823
8  гражданский брак                 1      F   сотрудник     0         95856
9   женат / замужем                 0      M   сотрудник     0        144425

                      purpose          purpose\_group  \textbackslash{}
0               покупка жилья                ипотека
1     приобретение автомобиля             автокредит
2               покупка жилья                ипотека
3  дополнительное образование  кредит на образование
4             сыграть свадьбу      кредит на свадьбу
5               покупка жилья                ипотека
6           операции с жильем                ипотека
7                 образование  кредит на образование
8       на проведение свадьбы      кредит на свадьбу
9     покупка жилья для семьи                ипотека

                income\_category
0  очень высокий уровень дохода
1        высокий уровень дохода
2        высокий уровень дохода
3  очень высокий уровень дохода
4        высокий уровень дохода
5  очень высокий уровень дохода
6  очень высокий уровень дохода
7        высокий уровень дохода
8        высокий уровень дохода
9        высокий уровень дохода
\end{Verbatim}
\end{tcolorbox}
        
    \begin{tcolorbox}[breakable, size=fbox, boxrule=1pt, pad at break*=1mm,colback=cellbackground, colframe=cellborder]
\prompt{In}{incolor}{35}{\hspace{4pt}}
\begin{Verbatim}[commandchars=\\\{\}]
\PY{c+c1}{\PYZsh{}теперь можно посмотреть сколько людей в каждой категории: 21471 из 21525}
\PY{n}{data}\PY{p}{[}\PY{l+s+s1}{\PYZsq{}}\PY{l+s+s1}{income\PYZus{}category}\PY{l+s+s1}{\PYZsq{}}\PY{p}{]}\PY{o}{.}\PY{n}{value\PYZus{}counts}\PY{p}{(}\PY{p}{)}
\end{Verbatim}
\end{tcolorbox}

            \begin{tcolorbox}[breakable, boxrule=.5pt, size=fbox, pad at break*=1mm, opacityfill=0]
\prompt{Out}{outcolor}{35}{\hspace{3.5pt}}
\begin{Verbatim}[commandchars=\\\{\}]
высокий уровень дохода          9581
очень высокий уровень дохода    6422
средний уровень дохода          3131
низкий уровень дохода           2337
Name: income\_category, dtype: int64
\end{Verbatim}
\end{tcolorbox}
        
    \begin{tcolorbox}[breakable, size=fbox, boxrule=1pt, pad at break*=1mm,colback=cellbackground, colframe=cellborder]
\prompt{In}{incolor}{36}{\hspace{4pt}}
\begin{Verbatim}[commandchars=\\\{\}]
\PY{c+c1}{\PYZsh{}Теперь посмотрим стобец \PYZdq{}children\PYZdq{}}
\PY{k}{def} \PY{n+nf}{children\PYZus{}category}\PY{p}{(}\PY{n}{children}\PY{p}{)}\PY{p}{:}
    \PY{k}{if} \PY{n}{children} \PY{o}{\PYZlt{}}\PY{o}{=} \PY{l+m+mi}{0}\PY{p}{:}
        \PY{k}{return} \PY{l+s+s1}{\PYZsq{}}\PY{l+s+s1}{нет детей}\PY{l+s+s1}{\PYZsq{}}
    \PY{k}{if} \PY{n}{children} \PY{o}{\PYZgt{}}\PY{o}{=} \PY{l+m+mi}{0}\PY{p}{:}
        \PY{k}{return} \PY{l+s+s1}{\PYZsq{}}\PY{l+s+s1}{есть дети}\PY{l+s+s1}{\PYZsq{}}

\PY{n}{data}\PY{p}{[}\PY{l+s+s1}{\PYZsq{}}\PY{l+s+s1}{children\PYZus{}category}\PY{l+s+s1}{\PYZsq{}}\PY{p}{]} \PY{o}{=} \PY{n}{data}\PY{p}{[}\PY{l+s+s1}{\PYZsq{}}\PY{l+s+s1}{children}\PY{l+s+s1}{\PYZsq{}}\PY{p}{]}\PY{o}{.}\PY{n}{apply}\PY{p}{(}\PY{n}{children\PYZus{}category}\PY{p}{)}
\PY{n}{data}\PY{o}{.}\PY{n}{head}\PY{p}{(}\PY{l+m+mi}{10}\PY{p}{)}
\end{Verbatim}
\end{tcolorbox}

            \begin{tcolorbox}[breakable, boxrule=.5pt, size=fbox, pad at break*=1mm, opacityfill=0]
\prompt{Out}{outcolor}{36}{\hspace{3.5pt}}
\begin{Verbatim}[commandchars=\\\{\}]
   children  days\_employed  dob\_years education  education\_id  \textbackslash{}
0         1   -8437.673028         42    высшее             0
1         1   -4024.803754         36   среднее             1
2         0   -5623.422610         33   Среднее             1
3         3   -4124.747207         32   среднее             1
4         0  340266.072047         53   среднее             1
5         0    -926.185831         27    высшее             0
6         0   -2879.202052         43    высшее             0
7         0    -152.779569         50   СРЕДНЕЕ             1
8         2   -6929.865299         35    ВЫСШЕЕ             0
9         0   -2188.756445         41   среднее             1

      family\_status  family\_status\_id gender income\_type  debt  total\_income  \textbackslash{}
0   женат / замужем                 0      F   сотрудник     0        253875
1   женат / замужем                 0      F   сотрудник     0        112080
2   женат / замужем                 0      M   сотрудник     0        145885
3   женат / замужем                 0      M   сотрудник     0        267628
4  гражданский брак                 1      F   пенсионер     0        158616
5  гражданский брак                 1      M   компаньон     0        255763
6   женат / замужем                 0      F   компаньон     0        240525
7   женат / замужем                 0      M   сотрудник     0        135823
8  гражданский брак                 1      F   сотрудник     0         95856
9   женат / замужем                 0      M   сотрудник     0        144425

                      purpose          purpose\_group  \textbackslash{}
0               покупка жилья                ипотека
1     приобретение автомобиля             автокредит
2               покупка жилья                ипотека
3  дополнительное образование  кредит на образование
4             сыграть свадьбу      кредит на свадьбу
5               покупка жилья                ипотека
6           операции с жильем                ипотека
7                 образование  кредит на образование
8       на проведение свадьбы      кредит на свадьбу
9     покупка жилья для семьи                ипотека

                income\_category children\_category
0  очень высокий уровень дохода         есть дети
1        высокий уровень дохода         есть дети
2        высокий уровень дохода         нет детей
3  очень высокий уровень дохода         есть дети
4        высокий уровень дохода         нет детей
5  очень высокий уровень дохода         нет детей
6  очень высокий уровень дохода         нет детей
7        высокий уровень дохода         нет детей
8        высокий уровень дохода         есть дети
9        высокий уровень дохода         нет детей
\end{Verbatim}
\end{tcolorbox}
        
    \begin{tcolorbox}[breakable, size=fbox, boxrule=1pt, pad at break*=1mm,colback=cellbackground, colframe=cellborder]
\prompt{In}{incolor}{37}{\hspace{4pt}}
\begin{Verbatim}[commandchars=\\\{\}]
\PY{c+c1}{\PYZsh{}теперь можно посмотреть сколько людей в каждой категории: 21471 из 21525}
\PY{n}{data}\PY{p}{[}\PY{l+s+s1}{\PYZsq{}}\PY{l+s+s1}{children\PYZus{}category}\PY{l+s+s1}{\PYZsq{}}\PY{p}{]}\PY{o}{.}\PY{n}{value\PYZus{}counts}\PY{p}{(}\PY{p}{)}
\end{Verbatim}
\end{tcolorbox}

            \begin{tcolorbox}[breakable, boxrule=.5pt, size=fbox, pad at break*=1mm, opacityfill=0]
\prompt{Out}{outcolor}{37}{\hspace{3.5pt}}
\begin{Verbatim}[commandchars=\\\{\}]
нет детей    14107
есть дети     7364
Name: children\_category, dtype: int64
\end{Verbatim}
\end{tcolorbox}
        
    \hypertarget{ux432ux44bux432ux43eux434}{%
\subsubsection{Вывод}\label{ux432ux44bux432ux43eux434}}

    Разбили на категории два столбца, которые необходимы нам для дальнейшего
анализа. Применили методы value\_counts(), apply(), def-if-retutn.

    Общий вывод по ПЕРЕРАБОТКЕ ДАННЫХ: 1. обработали пропуска 2. удалили
дубликаты 3. выделили леммы в столбце ``purpose'' 4. категоризировали
данные

    Комментарий наставника

Категоризация проведена верно

    \hypertarget{ux448ux430ux433-3.-ux43eux442ux432ux435ux442ux44cux442ux435-ux43dux430-ux432ux43eux43fux440ux43eux441ux44b}{%
\subsubsection{Шаг 3. Ответьте на
вопросы}\label{ux448ux430ux433-3.-ux43eux442ux432ux435ux442ux44cux442ux435-ux43dux430-ux432ux43eux43fux440ux43eux441ux44b}}

    \begin{itemize}
\tightlist
\item
  Есть ли зависимость между наличием детей и возвратом кредита в срок?
  Гипотеза: зависимость между наличием детей и возвратом кредита в срок
  есть, люди с детьми отдают кредит с большей вероятностью во время.
\end{itemize}

    \begin{tcolorbox}[breakable, size=fbox, boxrule=1pt, pad at break*=1mm,colback=cellbackground, colframe=cellborder]
\prompt{In}{incolor}{38}{\hspace{4pt}}
\begin{Verbatim}[commandchars=\\\{\}]
\PY{c+c1}{\PYZsh{}Для ответа на вопрос нам потребуется рассмотреть два столбца: children\PYZus{}category и debt. }
\PY{n}{children\PYZus{}pivot} \PY{o}{=} \PY{n}{data}\PY{o}{.}\PY{n}{pivot\PYZus{}table}\PY{p}{(}\PY{n}{index} \PY{o}{=} \PY{p}{[}\PY{l+s+s1}{\PYZsq{}}\PY{l+s+s1}{children\PYZus{}category}\PY{l+s+s1}{\PYZsq{}}\PY{p}{]}\PY{p}{,} \PY{n}{columns} \PY{o}{=} \PY{l+s+s1}{\PYZsq{}}\PY{l+s+s1}{debt}\PY{l+s+s1}{\PYZsq{}}\PY{p}{,} \PY{n}{values} \PY{o}{=} \PY{l+s+s1}{\PYZsq{}}\PY{l+s+s1}{purpose\PYZus{}group}\PY{l+s+s1}{\PYZsq{}}\PY{p}{,} \PY{n}{aggfunc}\PY{o}{=}\PY{l+s+s1}{\PYZsq{}}\PY{l+s+s1}{count}\PY{l+s+s1}{\PYZsq{}}\PY{p}{)}
\PY{n}{children\PYZus{}pivot}\PY{p}{[}\PY{l+s+s1}{\PYZsq{}}\PY{l+s+s1}{ratio}\PY{l+s+s1}{\PYZsq{}}\PY{p}{]} \PY{o}{=} \PY{n}{children\PYZus{}pivot}\PY{p}{[}\PY{l+m+mi}{1}\PY{p}{]} \PY{o}{/} \PY{n}{children\PYZus{}pivot}\PY{p}{[}\PY{l+m+mi}{0}\PY{p}{]} \PY{o}{*} \PY{l+m+mi}{100}
\PY{n}{children\PYZus{}pivot}
\PY{n+nb}{print}\PY{p}{(}\PY{n}{children\PYZus{}pivot}\PY{p}{)}
\end{Verbatim}
\end{tcolorbox}

    \begin{Verbatim}[commandchars=\\\{\}]
debt                   0     1      ratio
children\_category
есть дети           6686   678  10.140592
нет детей          13044  1063   8.149341
\end{Verbatim}

    \hypertarget{ux432ux44bux432ux43eux434}{%
\subsubsection{Вывод}\label{ux432ux44bux432ux43eux434}}

Мы видим, что люди, у которых нет детей, отдают кредиты хуже, нежели,
чем люди с детьми (неважно какое количество детей). Возможно, потому что
люди с детьми более обязательные с повышенным чувством ответственности.
Гипотеза подтвердилась.

    \begin{itemize}
\tightlist
\item
  Есть ли зависимость между семейным положением и возвратом кредита в
  срок? Гипотеза: семейные люди с большей вероятностью отдадут кредит в
  срок, нежели люди без семьи.
\end{itemize}

    \begin{tcolorbox}[breakable, size=fbox, boxrule=1pt, pad at break*=1mm,colback=cellbackground, colframe=cellborder]
\prompt{In}{incolor}{39}{\hspace{4pt}}
\begin{Verbatim}[commandchars=\\\{\}]
\PY{c+c1}{\PYZsh{}рассмотрим два столбца: family\PYZus{}status и debt}
\PY{n}{family\PYZus{}status\PYZus{}pivot} \PY{o}{=} \PY{n}{data}\PY{o}{.}\PY{n}{pivot\PYZus{}table}\PY{p}{(}\PY{n}{index} \PY{o}{=} \PY{p}{[}\PY{l+s+s1}{\PYZsq{}}\PY{l+s+s1}{family\PYZus{}status}\PY{l+s+s1}{\PYZsq{}}\PY{p}{]}\PY{p}{,} \PY{n}{columns} \PY{o}{=} \PY{l+s+s1}{\PYZsq{}}\PY{l+s+s1}{debt}\PY{l+s+s1}{\PYZsq{}}\PY{p}{,} \PY{n}{values} \PY{o}{=} \PY{l+s+s1}{\PYZsq{}}\PY{l+s+s1}{purpose\PYZus{}group}\PY{l+s+s1}{\PYZsq{}}\PY{p}{,} \PY{n}{aggfunc}\PY{o}{=}\PY{l+s+s1}{\PYZsq{}}\PY{l+s+s1}{count}\PY{l+s+s1}{\PYZsq{}}\PY{p}{)}
\PY{n}{family\PYZus{}status\PYZus{}pivot}\PY{p}{[}\PY{l+s+s1}{\PYZsq{}}\PY{l+s+s1}{ratio}\PY{l+s+s1}{\PYZsq{}}\PY{p}{]} \PY{o}{=} \PY{n}{family\PYZus{}status\PYZus{}pivot}\PY{p}{[}\PY{l+m+mi}{1}\PY{p}{]} \PY{o}{/} \PY{n}{family\PYZus{}status\PYZus{}pivot}\PY{p}{[}\PY{l+m+mi}{0}\PY{p}{]} \PY{o}{*} \PY{l+m+mi}{100}
\PY{n}{family\PYZus{}status\PYZus{}pivot}
\PY{n+nb}{print}\PY{p}{(}\PY{n}{family\PYZus{}status\PYZus{}pivot}\PY{p}{)}
\end{Verbatim}
\end{tcolorbox}

    \begin{Verbatim}[commandchars=\\\{\}]
debt                       0    1      ratio
family\_status
в разводе               1110   85   7.657658
вдовец / вдова           896   63   7.031250
гражданский брак        3775  388  10.278146
женат / замужем        11413  931   8.157364
не женат / не замужем   2536  274  10.804416
\end{Verbatim}

    \hypertarget{ux432ux44bux432ux43eux434}{%
\subsubsection{Вывод}\label{ux432ux44bux432ux43eux434}}

    Мы наблюдаем, что самые низкие показатели по возврату кредита в срок у
тех, кто в разводе и вдов/вдовцов. Возможно, потому что отсуствует
кормилец в семье, меньше общий доход (так как, когда два человека в
семье зарабатывают, вероятность возврата кредита в срок гораздо выше).
Гипотеза подтвердилась.

    \begin{itemize}
\tightlist
\item
  Есть ли зависимость между уровнем дохода и возвратом кредита в срок?
  Гипотеза: люди с высоким уровнем дохода с большей вероятностью будут
  возвращать кредит в срок, нежели люди с низким уровнем дохода.
\end{itemize}

    \begin{tcolorbox}[breakable, size=fbox, boxrule=1pt, pad at break*=1mm,colback=cellbackground, colframe=cellborder]
\prompt{In}{incolor}{40}{\hspace{4pt}}
\begin{Verbatim}[commandchars=\\\{\}]
\PY{n}{income\PYZus{}category\PYZus{}pivot} \PY{o}{=} \PY{n}{data}\PY{o}{.}\PY{n}{pivot\PYZus{}table}\PY{p}{(}\PY{n}{index} \PY{o}{=} \PY{p}{[}\PY{l+s+s1}{\PYZsq{}}\PY{l+s+s1}{income\PYZus{}category}\PY{l+s+s1}{\PYZsq{}}\PY{p}{]}\PY{p}{,} \PY{n}{columns} \PY{o}{=} \PY{l+s+s1}{\PYZsq{}}\PY{l+s+s1}{debt}\PY{l+s+s1}{\PYZsq{}}\PY{p}{,} \PY{n}{values} \PY{o}{=} \PY{l+s+s1}{\PYZsq{}}\PY{l+s+s1}{purpose\PYZus{}group}\PY{l+s+s1}{\PYZsq{}}\PY{p}{,} \PY{n}{aggfunc}\PY{o}{=}\PY{l+s+s1}{\PYZsq{}}\PY{l+s+s1}{count}\PY{l+s+s1}{\PYZsq{}}\PY{p}{)}
\PY{n}{income\PYZus{}category\PYZus{}pivot}\PY{p}{[}\PY{l+s+s1}{\PYZsq{}}\PY{l+s+s1}{ratio}\PY{l+s+s1}{\PYZsq{}}\PY{p}{]} \PY{o}{=} \PY{n}{income\PYZus{}category\PYZus{}pivot}\PY{p}{[}\PY{l+m+mi}{1}\PY{p}{]} \PY{o}{/} \PY{n}{income\PYZus{}category\PYZus{}pivot}\PY{p}{[}\PY{l+m+mi}{0}\PY{p}{]} \PY{o}{*} \PY{l+m+mi}{100}
\PY{n}{income\PYZus{}category\PYZus{}pivot}
\PY{n+nb}{print}\PY{p}{(}\PY{n}{income\PYZus{}category\PYZus{}pivot}\PY{p}{)}
\end{Verbatim}
\end{tcolorbox}

    \begin{Verbatim}[commandchars=\\\{\}]
debt                             0    1     ratio
income\_category
высокий уровень дохода        8750  831  9.497143
низкий уровень дохода         2153  184  8.546215
очень высокий уровень дохода  5944  478  8.041723
средний уровень дохода        2883  248  8.602151
\end{Verbatim}

    \hypertarget{ux432ux44bux432ux43eux434}{%
\subsubsection{Вывод}\label{ux432ux44bux432ux43eux434}}

    Есть несущественная разница влияния уровня дохода на возврат кредита в
срок. Стоит отметить, что хуже всего на возврат кредита в срок
сказывается очень высокий уровень дохода, на втором месте идет - низкий
уровень дохода. Возможно, поэтому уровень дохода очень высокий, потому
что не возвращают кредит в срок =). Почему люди с низким уровнем дохода
возвращают кредит в срок хуже, в целом объяснимо. Гипотеза подтвердилась
наполовину.

    \begin{itemize}
\tightlist
\item
  Как разные цели кредита влияют на его возврат в срок? Гипотеза: цели
  влияют на кредит возврата в срок.
\end{itemize}

    \begin{tcolorbox}[breakable, size=fbox, boxrule=1pt, pad at break*=1mm,colback=cellbackground, colframe=cellborder]
\prompt{In}{incolor}{41}{\hspace{4pt}}
\begin{Verbatim}[commandchars=\\\{\}]
\PY{n}{purpose\PYZus{}group\PYZus{}pivot} \PY{o}{=} \PY{n}{data}\PY{o}{.}\PY{n}{pivot\PYZus{}table}\PY{p}{(}\PY{n}{index} \PY{o}{=} \PY{p}{[}\PY{l+s+s1}{\PYZsq{}}\PY{l+s+s1}{purpose\PYZus{}group}\PY{l+s+s1}{\PYZsq{}}\PY{p}{]}\PY{p}{,} \PY{n}{columns} \PY{o}{=} \PY{l+s+s1}{\PYZsq{}}\PY{l+s+s1}{debt}\PY{l+s+s1}{\PYZsq{}}\PY{p}{,} \PY{n}{values} \PY{o}{=} \PY{l+s+s1}{\PYZsq{}}\PY{l+s+s1}{total\PYZus{}income}\PY{l+s+s1}{\PYZsq{}}\PY{p}{,} \PY{n}{aggfunc}\PY{o}{=}\PY{l+s+s1}{\PYZsq{}}\PY{l+s+s1}{count}\PY{l+s+s1}{\PYZsq{}}\PY{p}{)}
\PY{n}{purpose\PYZus{}group\PYZus{}pivot}\PY{p}{[}\PY{l+s+s1}{\PYZsq{}}\PY{l+s+s1}{ratio}\PY{l+s+s1}{\PYZsq{}}\PY{p}{]} \PY{o}{=} \PY{n}{purpose\PYZus{}group\PYZus{}pivot}\PY{p}{[}\PY{l+m+mi}{1}\PY{p}{]} \PY{o}{/} \PY{n}{purpose\PYZus{}group\PYZus{}pivot}\PY{p}{[}\PY{l+m+mi}{0}\PY{p}{]} \PY{o}{*} \PY{l+m+mi}{100}
\PY{n}{purpose\PYZus{}group\PYZus{}pivot}
\PY{n+nb}{print}\PY{p}{(}\PY{n}{purpose\PYZus{}group\PYZus{}pivot}\PY{p}{)}
\end{Verbatim}
\end{tcolorbox}

    \begin{Verbatim}[commandchars=\\\{\}]
debt                       0    1      ratio
purpose\_group
автокредит              3905  403  10.320102
ипотека                10032  782   7.795056
кредит на образование   3644  370  10.153677
кредит на свадьбу       2149  186   8.655188
\end{Verbatim}

    \hypertarget{ux432ux44bux432ux43eux434}{%
\subsubsection{Вывод}\label{ux432ux44bux432ux43eux434}}

    Наименьшие показатели возврата кредита в срок мы видим у следующих
целей: ипотека и кредит на свадьбу. Кредит на свадьбу - можно
охарактеризовать как потребительский, скорее всего его к нему относятся
не совсем серьезно, да и на свадьбу брать кредит - не очень серьезная
история. Ипотека - достаточно рисковая сделка, большие вложения,
которыми, скорее всего, не всегда располагают люди, берущие такого рода
кредит.

    Комментарий наставника

Все выводы верны, код написан правильно, радует, что используешь сводные
таблицы)

    \hypertarget{ux448ux430ux433-4.-ux43eux431ux449ux438ux439-ux432ux44bux432ux43eux434}{%
\subsubsection{Шаг 4. Общий
вывод}\label{ux448ux430ux433-4.-ux43eux431ux449ux438ux439-ux432ux44bux432ux43eux434}}

    Результаты исследования показали, что мы наблюдаем зависимость между
семейным положением, наличием детей, уровнем дохода и целями на кредит
на возврат кредита в срок. - Наиболее повышенны риски по невозврату
кредитов наблюдаются у вдовцов/вдов и у разведенных, скорее всего
потому, что люди оказались в непростой ситуации, отсутсвие кормильца и
т.д. - Также следует обратить внимание на цели кредита, а именно кредит
на свадьбу и ипотеку. Ипотека - очень крупная сделка с банком с больщим
количествои рисков, а кредит на свадьбу - не очень ликвидный кредит. -
Наличие детей влияет не кардинально на возврат кредита в срок, это не
тот пункт на котором стоит заострять внимание. - Уровень дохода также
влияет на возврат кредита в срок.

    Комментарий наставника

Хороший вывод, как совет на будущее - можешь в выводе писать
рекомендации по использованию данного анализа

    \hypertarget{ux447ux435ux43a-ux43bux438ux441ux442-ux433ux43eux442ux43eux432ux43dux43eux441ux442ux438-ux43fux440ux43eux435ux43aux442ux430}{%
\subsubsection{Чек-лист готовности
проекта}\label{ux447ux435ux43a-ux43bux438ux441ux442-ux433ux43eux442ux43eux432ux43dux43eux441ux442ux438-ux43fux440ux43eux435ux43aux442ux430}}

Поставьте `x' в выполненных пунктах. Далее нажмите Shift+Enter.

    \begin{itemize}
\tightlist
\item[$\boxtimes$]
  открыт файл;
\item[$\boxtimes$]
  файл изучен;
\item[$\boxtimes$]
  определены пропущенные значения;
\item[$\boxtimes$]
  заполнены пропущенные значения;
\item[$\boxtimes$]
  есть пояснение, какие пропущенные значения обнаружены;
\item[$\boxtimes$]
  описаны возможные причины появления пропусков в данных;
\item[$\boxtimes$]
  объяснено, по какому принципу заполнены пропуски;
\item[$\boxtimes$]
  заменен вещественный тип данных на целочисленный;
\item[$\boxtimes$]
  есть пояснение, какой метод используется для изменения типа данных и
  почему;
\item[$\boxtimes$]
  удалены дубликаты;
\item[$\boxtimes$]
  есть пояснение, какой метод используется для поиска и удаления
  дубликатов;
\item[$\boxtimes$]
  описаны возможные причины появления дубликатов в данных;
\item[$\boxtimes$]
  выделены леммы в значениях столбца с целями получения кредита;
\item[$\boxtimes$]
  описан процесс лемматизации;
\item[$\boxtimes$]
  данные категоризированы;
\item[$\boxtimes$]
  есть объяснение принципа категоризации данных;
\item[$\boxtimes$]
  есть ответ на вопрос: ``Есть ли зависимость между наличием детей и
  возвратом кредита в срок?'';
\item[$\boxtimes$]
  есть ответ на вопрос: ``Есть ли зависимость между семейным положением
  и возвратом кредита в срок?'';
\item[$\boxtimes$]
  есть ответ на вопрос: ``Есть ли зависимость между уровнем дохода и
  возвратом кредита в срок?'';
\item[$\boxtimes$]
  есть ответ на вопрос: ``Как разные цели кредита влияют на его возврат
  в срок?'';
\item[$\boxtimes$]
  в каждом этапе есть выводы;
\item[$\boxtimes$]
  есть общий вывод.
\end{itemize}

    Комментарий наставника

\hypertarget{ux43aux43eux434}{%
\paragraph{Код}\label{ux43aux43eux434}}

Всё отлично. Из того, что очень порадовало - соблюдена структура
проекта, шаги из задания обозначены и выполнены последовательно, код
написан аккуратно, используются комментарии к коду, быстро можно понять,
какие операции выполняют сложные конструкции. В качестве совета
предлагаю глубже изучить и начать чаще применять конструкцию try-except
в решении задачи --- это улучшит отказоустойчивость кода и обезопасит
код от поломок в будущем, а так же изучить средства для построения
изображений в Python для более глубокого понимания данных. \#\#\#\#
Выводы

У тебя отлично получается анализировать сложные данные, выдвигать
корректные гипотезы и проверять свои выводы на возможность соответствия
реальности. Видно глубокое понимание сути проведённого анализа. Было
очень интересно проверять твой проект и следить за твоей мыслью, так
держать!)

    Комментарий наставника

Работа принята!) Очень рад, что у тебя всё получилось и надеюсь тебе
понравится весь курс, будь усердна и у тебя всё получится! Удачи в
будущих проектах)

    \begin{tcolorbox}[breakable, size=fbox, boxrule=1pt, pad at break*=1mm,colback=cellbackground, colframe=cellborder]
\prompt{In}{incolor}{ }{\hspace{4pt}}
\begin{Verbatim}[commandchars=\\\{\}]

\end{Verbatim}
\end{tcolorbox}


    % Add a bibliography block to the postdoc
    
    
    
    \end{document}
